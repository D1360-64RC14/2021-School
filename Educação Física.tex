\documentclass{SchoolBook}

\begin{document}
    \begin{day}{23/04/2021}
        \title{3}{Forró}
        
        \begin{itemize}[nosep]
            \item Origem: bailes populares;
            \item Câmara Cascudo: forrobodó;
            \item Eventos abertos ao público: \emph{"for all"};
            \item 1950: Luiz Gonzaga -- "Forró de Mané Vito";
            \vspace{3pt}
            \item Gênero musical: Xote, xaxado e baião;
            \item Forró tradicional ou Forró-pé-de-serra;
            \item Base instrumental: sanfona, triângulo e zabumba;
        \end{itemize}
    \end{day}
    
    \begin{day}{26/04/2021}
        \title{3}{O que vimos na aula}
    
        Forró é a festa onde se toca e dança gêneros musicais nordestinos, tais como o boião, o xote, o xaxado, o coco e a quadrilha.
        
        Também são gêneros musicas: xote, xaxado e baião;
        
        Forró tradicional ou Forró pé-de-serra.
        
        Base instrumental: sanfona, triângulo e zabumba.
        
        O forró evoluiu para os gêneros eletrônico e universitário.
    \end{day}
    
    \begin{day}{30/04/2021}
        \title{2}{Ginástica de Condicionamento Físico}
        \title{3}{Produção histórica em ginástica}
        
        \textbf{Ginástica}: prática de atividade física;
        
        \textbf{Pré-história}: possuía um caráter natural, utilitário, guerreiro, ritualistico e recreativo;
        
        \textbf{Antiguidade}: objeto de culto, recreação e preparação guerreira.
        
        \vspace{3pt}
        
        \textbf{Era clássica}:
        
        \qquad\underline{Grécia}: finalidade educacional para formar cidadãos e também guerreiros;
        
        \qquad\underline{Império Romano}: formação de guerreiros;
        
        \qquad\underline{Idade Média}: cultivar a espiritualidade.
    \end{day}
    
    \begin{day}{07/05/2021}
        \title{3}{Conceitos}
        
        \begin{itemize}
            \item \textbf{Atividade físia}: \textit{qualquer} movimento corporal;
            \item \textbf{Exercício físico}: \textit{planejado, estruturado e repetitivo} -- condicionamento físico;
            \item \textbf{Ginástica}: \textit{prática} de exercício físico.
        \end{itemize}
        
        \textbf{Condicionamento Físico}: melhora no funcionamento músculo-esquelético e metabólico -- força muscular, potência, resistência cardiovascular, resistência muscula e na flexibilidade.
        
        \textbf{Ginástica de Condicionamento Físico}: uma forma de se adquirir ou manter a saúde.
        
        \textbf{Ginástica de Competição}: regras pré-estabelecidas intencionalmente.
        
        \textbf{Ginástica Artística}: solo e aparelhos.
        
        \underline{Aparelhos Femininos}: solo, salto sobre a mesa, barras assimétricas e trave.
        
        \underline{Aparelhos Masculinos}: solo, salto sobre a mesa, cavalo com alças, barras paralelas, barra fixa e argolas.
    \end{day}
    
    \begin{day}{10/05/2021}
        \title{3}{Ginásica Acrobática}
        
        \title{2}{Tipos de Ginásticas}
        
        \title{3}{\underline{Ginásticas não competitivas}}
        
        \begin{itemize}[nosep]
            \item \textbf{Ginásticas de Conscientização Corporal}: soluções para problemas de saúde e posturas;
            \item \textbf{Ginásticas Fisioterápicas}: prevenção e tratamento de doenças;
            \item \textbf{Ginástica Laboral}: trabalho;
            \item \textbf{Ginástica de Considiconamento Físico}: musculação, \textit{step}, zumba, danças, lutas, \textit{spinning}, etc.
        \end{itemize}
        
        \title{3}{Revisando}
        
        A ginástica foi produzida em diferentes períodos e contextos históricos;
        
        Ginástiva de condicionamento físico: saúde, "corpo ideal" e mundo do trabalho;
        
        Aprovação social e sobrevivência;
        
        Esportivização das ginásticas: disciplica, rencimento, comparação de resultados, busca pela vitória e padronização dos corpos;
        
        Modismos e consumo.
    \end{day}
    
    \begin{day}{2º trimestre --- 24/05/2021}
        \title{3}{Vida Sedentária Versus Qualidade de Vida}
        \textbf{Sedentarismo}: condição de não atingir as diretrizes da saúde pública para os níveis recomendados de atividade física de intensidade moderada e vigorosa; qualquer comportamento de vigília enquanto estiver sentado, reclinado ou deitado.
        
        \title{3}{Sobre o Sedentarismo}
        
        \begin{itemize}[nosep]
            \item Até meados de 1990 houve mudança nos níveis de aptidão física das pessoas.
            \item Atividade física vs. doenças.
            \item Os avanços tecnológicos criaram comportamentos sedentários.
        \end{itemize}
        
        \title{3}{O que é qualidade de vida?}
        
        Bem-estar geral, no tempo de trabalho e de não trabalho, acesso à saúde, à educação, ao ócio, às artes, atividade física, etc.
    \end{day}
    
    \begin{day}{28/05/2021}
        \title{3}{Classificação das ginásticas segundo a BNCC}
        
        \begin{itemize}[nosep]
            \item Ginástica Geral $\Longrightarrow$ Ginástica para todos
            \item Ginástica de Condicionamento Físico
            \item Ginástica de Conscientização Corporal
            \item Ginástica de Competição
        \end{itemize}
    \end{day}
    
    \begin{day}{31/05/2021}
        \title{1}{Jogos}
        \title{3}{Conceito e Classificação dos Jogos}
        
        \begin{itemize}[nosep]
           %\item Jogo: brincadeira com regras dinâmicas.
            \item Diversão, porém com regras pré-definidas, que podem ser alteradas.
           %\item Esporte: jogo com regras sólidas e rígidas.
            \item "Sem diversão", apenas competição com regras universais definidas por instituições.
        \end{itemize}
        
        \title{3}{CLASSIFICAÇÃO}
        
        Existem várias maneirs, propostas por diversos autores, de se classificar os Jogos.
        
        Porém, iremos classifica-los da seguinte forma:
        
        \begin{itemize}[nosep]
            \item Jogos Populares;
            \item Jogos Dramáticos;
            \item Jogos de Tabuleiro;
            \item Jogos Cooperativos;
            \item Jogos Esportivos;
            \item Jogos Eletrônicos.
        \end{itemize}
    \end{day}
    
    \begin{day}{07/06/2021}
        \title{2}{Atividade Física e Exercício Físico}
        
        \noindent\textbf{Atividade física}: qualquer movimento corporal voluntário que resulte num gasto energético acima dos níveis de respouso.
        
        \noindent\textbf{Exercícios físicos}: subgrupo das atividades físicas, que é planejado, estruturado e repetitivo, tendo como propósito a manutenção ou a otimização do condicionamento físico.
        
        \title{2}{Atividade Aeróbica}
        
        Os grandes músculos do corpo se movem de maneira rítmica por um período prolongado de tempo. Um breve passeio, correr, andar de bicicleta, pular corda e nadar sã otodos exemplos.
        
        \title{4}{3 componentes}
        \vspace{6pt}
        \begin{itemize}[nosep]
            \item Intensidade: leve, moderada ou vigorosa;
            \item Frequência: vezes por semana;
            \item Duração: tempo em uma sessão.
        \end{itemize}
        
        \title{3}{Fortalecimento Muscular}
        
        \begin{itemize}
            \item Intensidade: leve, moderada ou vigorosa;
            \item frequência: vezes por semana;
            \item Séries: grupos de repetições;
            \item Repetições: execução de movimentos concêntricos e excêntricos.
        \end{itemize}
        
        \title{3}{Fortalecimento Ósseo}
        
        Essas atividades promovem uma sobrecarda nos ossos, gerando crescimento e força óssea.
        
        \vspace{12pt}
        \begin{itemize}[nosep]
            \item Polichinelo;
            \item Corrida;
            \item Caminhada acelerada;
            \item Exercícios de levantamente de peso.
        \end{itemize}
        
        Podem ser atividades aeróbicas e de fortalecimento muscular.
    \end{day}
    
    \begin{day}{11/06/2021}
        \title{3}{Introdução à Ginástica de Condicionamento Físico}
        
        \textbf{Aptidão Física}: capacidade que uma pessoa tem de realizar a Atividade Física de maneira satisfatória
        
        \textbf{Condicionamento Físico}: melhora no funcionamento muscolesquelético e metabólico -- força muscular, potência, resistência cardiovascular, resistência muscular e na flexibilidade.
        
        \textbf{Capacidades Físicas}: são as qualidades inatas, esto é, que nascem com o indivíduo e indicam uma medida de potencial ou talento que podem ser modelado e treinado. \emph{(Força, Resistência, flexibilidade, velocidade, ritmo, agilidade, coordenação, equilíbrio)}
        
        \title{3}{Ginástica de Condicionamento Físico}
        
        Exemplos: Ginástica Funcional, Ginástica Laboral, Crossfit, Fisiculturismo, Musculação, Hidroginástica, Alongamentos, Ginástica Aeróbica, Musculação, Pilates, Caminhada, Step, Jump.
    \end{day}
    
    \begin{day}{18/06/2021}
        \title 3 {O que são Fontes energéticas?}
        
        O corpo humano precisa de energia para manter suas funções vitais. Essas fontes energéticas fornecem a energia necessária para manter as funções corporais tanto em repouso quanto durante a modalidade praticada.
        
        As principais fontes energéticas são:
        
        \begin{itemize}[nosep]
            \item \textbf{Carboidratos} ou açúcares;
            \item \textbf{Proteína} (aminoácidos);
            \item \textbf{Lipídios} (gorduras).
        \end{itemize}
        
        \begin{center}
            Cite 3 exemplos de alimentos para cada fonte energética
            
            \vspace{6pt}
            \begin{tabular}{c|c|c}
                \textbf{Carboidratos} & \textbf{Proteínas} & \textbf{Lipídios} \\
                \hline
                Arroz Integral        & Carnes             & Azeite de Oliva   \\
                \hline
                Vegetais              & Ovo                & Castanha          \\
                \hline
                Frutas                & Leite e Derivados  & Carne Bovina      \\
                \hline
            \end{tabular}
        \end{center}
    \end{day}
    
    \begin{day}{21/06/2021}
        \title 3 {Fontes energéticas e o Esporte}
        
        \begin{itemize}
            \item Anaeríbico alático: intensidade máxima e curta duração;
            \item Anaeróbico lático: intensidade máxima e curta duração com produção de lactato;
            \item Aeróbico: não há esforço máximo e longa duração.
        \end{itemize}
        
        \begin{enumerate}
            \item[1.] Preencher a tabela abaixo, com uma modalidade esportiva para cada tipo de sistema de produção de energia, na qual essa modalidade utilize tal sistema de modo predominante.
            
            \vspace{6pt}
            \begin{tabular}{c|c|c}
                \textbf{Anaeróbico Alático} & \textbf{Anaeróbico Lático} & \textbf{Aeróbico} \\
                \hline
                100 metros rasos            & Corrida 800 metros         & Maratona \\
                \hline
            \end{tabular}
        \end{enumerate}
    \end{day}
    
    \begin{day}{25/06/2021}
        \title 3 {Alterações no Organismo em Resposta ao Exercício Físico}
        
        \begin{itemize}
            \item Frequência Cardíaca (FC): corresponte ao número de atimentos cardíacos por unidade de tempo (BPM). Serve como indicador de intensidade de exercício.
            \item Frequência Respiratória
            \item Temperatura Corporal
            \item Adaptaçẽos Musculares
            \item Resposta Imunológica
            \item Respostas Endócrinas: integra e regula as funções corporais, proporcionando estabilidade ao organismo em estados de repouso e de exercício.
        \end{itemize}
    \end{day}
    
    \begin{day}{05/07/2021}
        \title 3 {Força}
        
        Força muscular é a capacidade que o músculo tem de gerar tensão para superar ou se opor a uma resistência.
        
        Superação de uma dada resistência a qual ocorre por meio da conteação muscular.
        
        \title 3 {Resistência Muscular}
        
        É a capacidade do músculo para exercer repetidamente a força contra a resistência.
        
        Se seus músculos têm que contrair em um exercício mais de uma vez você está usando a resistência muscular.
        
        \title 3 {Potência Muscular}
        
        É o produto da força pela velocidade e se traduz na capacidade do músculo produzir força rapidamente.
        
        \title 3 {Atividade}
        
        Preencha a tabela abaixo com exemplos de atividades que exijam \textbf{força}, \textbf{resistência} e \textbf{potência} muscular:
        
        \begin{center}
            \begin{tabular}{c|c|c}
                \textbf{Força}        & \textbf{Resistência} & \textbf{Potência} \\
                \hline
                Lançamento de martelo & Escalada             & Basebal (arremessador)
            \end{tabular}
        \end{center}
    \end{day}
    
    \begin{day}{26/07/2021}
        \title 3 {Métodos de Treino}
        
        \begin{itemize}[nosep]
            \item Treino de resistência muscular;
            \item Treino de hipertrofia muscular;
            \item Treino de força muscular máxima ou pura;
            \item Treino de potência muscular.
        \end{itemize}
    \end{day}
    
    \begin{day}{26/07/2021}
        \title 3 {Os Riscos do Uso de Suplementos Alimentares e de Anabolizantes}
        
        \textbf{Suplementos alimentares}: produto para ingestão oral, apresentado em formas farmacêuticas, destinado a \underline{suplementar} a alimentação de indivíduos saudáveis com nutrientes, substâncias bioativs, enzimas ou probióticos, isolados ou combinados.
        
        \textbf{Anabolizantes}: também chamados de hormônios esteróides anabólicos androgênicos (EAA), esteróides anabolizantes, anabolizantes ou, simplesmente, \textit{"bombas"} (na linguagem popular), são substâncias sintéticas que tem como base a testosterona.
    \end{day}
    
    \begin{day}{30/07/2021}
        \title 3 {O que é Mídia?}
        
        Toda estrutura de difusão de informações, notícias, mensagens e entretenimento que estabece um canal intermediário de comunicação não pessoal, de comunicação de massa, utilizando-se de vários meios, entre eles jornais, revistas, rádio, televisão, cinema, mala direta, outdoors, informativos, telefone, internet, etc.
    \end{day}
    
    \begin{day}{02/08/2021}
        \title 3 {Imagem Corporal e Atividade Física}
        
        A \textbf{imagem corporal} é entendida como as percepções, pensamentos e sentimentos de um indivíduo sobre sua forma física.
    \end{day}
\end{document}
