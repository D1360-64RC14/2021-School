\documentclass{SchoolBook}

\begin{document}
    \begin{day}{23/04/2021}
        \title{3}{Forró}
        
        \begin{itemize}[nosep]
            \item Origem: bailes populares;
            \item Câmara Cascudo: forrobodó;
            \item Eventos abertos ao público: \emph{"for all"};
            \item 1950: Luiz Gonzaga -- "Forró de Mané Vito";
            \vspace{3pt}
            \item Gênero musical: Xote, xaxado e baião;
            \item Forró tradicional ou Forró-pé-de-serra;
            \item Base instrumental: sanfona, triângulo e zabumba;
        \end{itemize}
    \end{day}
    
    \begin{day}{26/04/2021}
        \title{3}{O que vimos na aula}
    
        Forró é a festa onde se toca e dança gêneros musicais nordestinos, tais como o boião, o xote, o xaxado, o coco e a quadrilha.
        
        Também são gêneros musicas: xote, xaxado e baião;
        
        Forró tradicional ou Forró pé-de-serra.
        
        Base instrumental: sanfona, triângulo e zabumba.
        
        O forró evoluiu para os gêneros eletrônico e universitário.
    \end{day}
    
    \begin{day}{30/04/2021}
        \title{2}{Ginástica de Condicionamento Físico}
        \title{3}{Produção histórica em ginástica}
        
        \textbf{Ginástica}: prática de atividade física;
        
        \textbf{Pré-história}: possuía um caráter natural, utilitário, guerreiro, ritualistico e recreativo;
        
        \textbf{Antiguidade}: objeto de culto, recreação e preparação guerreira.
        
        \vspace{3pt}
        
        \textbf{Era clássica}:
        
        \qquad\underline{Grécia}: finalidade educacional para formar cidadãos e também guerreiros;
        
        \qquad\underline{Império Romano}: formação de guerreiros;
        
        \qquad\underline{Idade Média}: cultivar a espiritualidade.
    \end{day}
\end{document}
