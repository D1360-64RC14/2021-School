\documentclass{SchoolBook}

\begin{document}
    \begin{day}{23/04/2021}
        \title{3}{Forró}
        
        \begin{itemize}[nosep]
            \item Origem: bailes populares;
            \item Câmara Cascudo: forrobodó;
            \item Eventos abertos ao público: \emph{"for all"};
            \item 1950: Luiz Gonzaga -- "Forró de Mané Vito";
            \vspace{3pt}
            \item Gênero musical: Xote, xaxado e baião;
            \item Forró tradicional ou Forró-pé-de-serra;
            \item Base instrumental: sanfona, triângulo e zabumba;
        \end{itemize}
    \end{day}
    
    \begin{day}{26/04/2021}
        \title{3}{O que vimos na aula}
    
        Forró é a festa onde se toca e dança gêneros musicais nordestinos, tais como o boião, o xote, o xaxado, o coco e a quadrilha.
        
        Também são gêneros musicas: xote, xaxado e baião;
        
        Forró tradicional ou Forró pé-de-serra.
        
        Base instrumental: sanfona, triângulo e zabumba.
        
        O forró evoluiu para os gêneros eletrônico e universitário.
    \end{day}
    
    \begin{day}{30/04/2021}
        \title{2}{Ginástica de Condicionamento Físico}
        \title{3}{Produção histórica em ginástica}
        
        \textbf{Ginástica}: prática de atividade física;
        
        \textbf{Pré-história}: possuía um caráter natural, utilitário, guerreiro, ritualistico e recreativo;
        
        \textbf{Antiguidade}: objeto de culto, recreação e preparação guerreira.
        
        \vspace{3pt}
        
        \textbf{Era clássica}:
        
        \qquad\underline{Grécia}: finalidade educacional para formar cidadãos e também guerreiros;
        
        \qquad\underline{Império Romano}: formação de guerreiros;
        
        \qquad\underline{Idade Média}: cultivar a espiritualidade.
    \end{day}
    
    \begin{day}{07/05/2021}
        \title{3}{Conceitos}
        
        \begin{itemize}
            \item \textbf{Atividade físia}: \textit{qualquer} movimento corporal;
            \item \textbf{Exercício físico}: \textit{planejado, estruturado e repetitivo} -- condicionamento físico;
            \item \textbf{Ginástica}: \textit{prática} de exercício físico.
        \end{itemize}
        
        \textbf{Condicionamento Físico}: melhora no funcionamento músculo-esquelético e metabólico -- força muscular, potência, resistência cardiovascular, resistência muscula e na flexibilidade.
        
        \textbf{Ginástica de Condicionamento Físico}: uma forma de se adquirir ou manter a saúde.
        
        \textbf{Ginástica de Competição}: regras pré-estabelecidas intencionalmente.
        
        \textbf{Ginástica Artística}: solo e aparelhos.
        
        \underline{Aparelhos Femininos}: solo, salto sobre a mesa, barras assimétricas e trave.
        
        \underline{Aparelhos Masculinos}: solo, salto sobre a mesa, cavalo com alças, barras paralelas, barra fixa e argolas.
    \end{day}
    
    \begin{day}{10/05/2021}
        \title{3}{Ginásica Acrobática}
        
        \title{2}{Tipos de Ginásticas}
        
        \title{3}{\underline{Ginásticas não competitivas}}
        
        \begin{itemize}[nosep]
            \item \textbf{Ginásticas de Conscientização Corporal}: soluções para problemas de saúde e posturas;
            \item \textbf{Ginásticas Fisioterápicas}: prevenção e tratamento de doenças;
            \item \textbf{Ginástica Laboral}: trabalho;
            \item \textbf{Ginástica de Considiconamento Físico}: musculação, \textit{step}, zumba, danças, lutas, \textit{spinning}, etc.
        \end{itemize}
        
        \title{3}{Revisando}
        
        A ginástica foi produzida em diferentes períodos e contextos históricos;
        
        Ginástiva de condicionamento físico: saúde, "corpo ideal" e mundo do trabalho;
        
        Aprovação social e sobrevivência;
        
        Esportivização das ginásticas: disciplica, rencimento, comparação de resultados, busca pela vitória e padronização dos corpos;
        
        Modismos e consumo.
    \end{day}
    
    \begin{day}{2º trimestre --- 24/05/2021}
        \title{3}{Vida Sedentária Versus Qualidade de Vida}
        \textbf{Sedentarismo}: condição de não atingir as diretrizes da saúde pública para os níveis recomendados de atividade física de intensidade moderada e vigorosa; qualquer comportamento de vigília enquanto estiver sentado, reclinado ou deitado.
        
        \title{3}{Sobre o Sedentarismo}
        
        \begin{itemize}[nosep]
            \item Até meados de 1990 houve mudança nos níveis de aptidão física das pessoas.
            \item Atividade física vs. doenças.
            \item Os avanços tecnológicos criaram comportamentos sedentários.
        \end{itemize}
        
        \title{3}{O que é qualidade de vida?}
        
        Bem-estar geral, no tempo de trabalho e de não trabalho, acesso à saúde, à educação, ao ócio, às artes, atividade física, etc.
    \end{day}
    
    \begin{day}{28/05/2021}
        \title{3}{Classificação das ginásticas segundo a BNCC}
        
        \begin{itemize}[nosep]
            \item Ginástica Geral $\Longrightarrow$ Ginástica para todos
            \item Ginástica de Condicionamento Físico
            \item Ginástica de Conscientização Corporal
            \item Ginástica de Competição
        \end{itemize}
    \end{day}
    
    \begin{day}{31/05/2021}
        \title{1}{Jogos}
        \title{3}{Conceito e Classificação dos Jogos}
        
        \begin{itemize}[nosep]
           %\item Jogo: brincadeira com regras dinâmicas.
            \item Diversão, porém com regras pré-definidas, que podem ser alteradas.
           %\item Esporte: jogo com regras sólidas e rígidas.
            \item "Sem diversão", apenas competição com regras universais definidas por instituições.
        \end{itemize}
        
        \title{3}{CLASSIFICAÇÃO}
        
        Existem várias maneirs, propostas por diversos autores, de se classificar os Jogos.
        
        Porém, iremos classifica-los da seguinte forma:
        
        \begin{itemize}[nosep]
            \item Jogos Populares;
            \item Jogos Dramáticos;
            \item Jogos de Tabuleiro;
            \item Jogos Cooperativos;
            \item Jogos Esportivos;
            \item Jogos Eletrônicos.
        \end{itemize}
    \end{day}
\end{document}
