\documentclass{SchoolBook}

\usepackage{enumitem}

\begin{document}
    \begin{day}{22/04/2021}
        \title{3}{Leia o texto a seguir e responda às questões.}
        \title{4}{Turma da Mônica: Laços}
        
        Filme da turminha é exemplo de adaptação e garantia de agrado aos fãs

        Quando foi anunciada a produção de um live-action da Turma da Mônica, foi difícil não ficar com um pé atrás. A hesitação é normal, já que qualquer adaptação de obra literária é arriscada, mas no caso da turminha que existe nos quadrinhos desde o fim dos anos 50, a questão era mais pessoal para o público brasileiro. [...]

        Laços tem um ritmo único que combina adaptação de HQs e filme infantil, sem perder timing de piadas e seguindo uma estrutura de aventura juvenil. [...]

        Enquanto todos os aspectos técnicos de Laços são dignos de menção, existe algo que fez do filme o que ele é, e que sem ele o projeto poderia ruir por completo: o elenco. [...] Daniel Rezende escolheu mostrar seus personagens sem flashbacks de origem, e tomou a liberdade de inserir certos elementos de Lições, a HQ seguinte, neste primeiro filme. [...]

        Por fim, Turma da Mônica: Laços marca o cinema nacional como uma obra exemplar de técnica, carisma, e adaptação de um dos nossos maiores patrimônios. É delicioso testemunhar o comprometimento da produção com o longa, o que faz com que cada fã do público se sinta representado de alguma forma pelo trabalho de Daniel Rezende. A sensação de que um fã qualificado tomou algo tão querido nas mãos é gratificante.

        \title{3}{ATIVIDADE}

        \begin{enumerate}
            \item[1.] Do que trata o texto?
            \response Sobre um filme adaptado dos quadrinhos.
            %O texo se trata sobre o \emph{live-action} Turma da Mônica: Laços.

            \item[2.] Que tipo de informação apresenta?
            \response Informações técnicas e analíticas.
            %Quando foi anunciado, criação e um pouco da obra original, como foi feita, elenco, opinião do escritor.

            \item[3.] É um texto descritivo ou opinativo? Justifique.
            \response Ambos. Descrevem alguns aspectos sobre a obra, acrecidos da opinião do autor do texto.
            %Um pouco de ambos, no início o autor descreve um pouco da obra, e o final dá opinião sobre a obra.

            \item[4.] Há pontos positivos e/ou negativos? Encontre-os.
            \response Somente pontos positivos. Aspectos técnicos e escolhas da direção.
        \end{enumerate}

        \title{3}{Importante lembrar\dots}

        A resenha é um texto sucinto, cuja principal característica é tecer, de maneira breve, uma crítica sobre determinado assunto.

        A resenha ideal é composta não apenas pela crítica direta, mas também por momentos de descrição, e esses dois elementos devem estar em perfeito equilíbrio em seu texto.
        
        \vspace{3pt}
        A resenha difere-se de um resumo porquê?
        \vspace{3pt}

        \begin{enumerate}[nosep]
            \item[a)] apresenta informações técnicas.
            \item[b)] apresenta informações polêmicas.
        \bf \item[c)] apresenta análises e opiniões. \normalfont
            \item[d)] apresenta gráficos e tabelas.
        \end{enumerate}

        \title{3}{ATIVIDADE}
        Indique, abaixo, o que é FATO e o que é OPINIÃO:
        \vspace{3pt}

        \begin{enumerate}[nosep]
        \bf \item[1.] "Mas no caso da turminha que existe nos quadrinhos desde o fim dos anos 50." \normalfont
        \bf \item[2.] "A inspiração dos Cafaggi para a \emph{graphic novel} veio de filmes infanto-juvenis dos anos 80." \normalfont
            \item[3.] "Eu, pelo menos, nunca tinha visto aquela estratégia infalível contra a Mônica."
            \item[4.] "Todos os aspetos técnicos de Laços são dignos de menção."
        \end{enumerate}
    \end{day}
\end{document}