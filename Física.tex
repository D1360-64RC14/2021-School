\documentclass{SchoolBook}

\usepackage{lipsum}
\usepackage{tikz}
\usepackage{amsmath}
\usepackage{multicol}
\usepackage{icomma}

\begin{document}
    \begin{day}{22/04/2021}
        \begin{enumerate}
            \item[1.] Sobre um corpo de massa $ 8 kg $, inicialmente em repouso, age uma força constante $ \vec{F} = 80 N $, na direção do deslocamento.
            Determine o trabalho realizado pela foça nos primeiros 20 segundos de movimento.
            \begin{align*}
                        \tau &= 80 * \Delta S   \\ 
                    \Delta S &= 0,5 * a * 20^2  \\
                          80 &= 8 * a           \\
                \frac{80}{8} &= a               \\
                    10 m/s^2 &= a               \\
                    \Delta S &= 0,5 * 10 * 20^2 \\
                    \Delta S &= 2000 metros
            \end{align*}

            \item[2.] Um ponto material de massa $ 6 kg $ tem velocidade de $ 8 m/s $ quando sobre ele passa a agir uma força de intensidade $ 30 N $ na direção do movimento, durante $ 4 s $. Determine:
            \begin{enumerate}
                \item[a)] o deslocamento durante esses $ 4 s$;
                \begin{align*}
                        \Delta S &= 8 + 0,5 * a * 4^2 \\
                              30 &= 6 * a             \\
                    \frac{30}{6} &= a                 \\
                         5 m/s^2 &= a                 \\
                        \Delta S &= 8 + 0,5 * 5 * 4^2 \\
                        \Delta S &= 169 metros
                \end{align*}
                
                \item[b)] o trabalho realizado nesse deslocamento.
                \begin{align*}
                    \tau &= 30 * 168 \\
                    \tau &= 5040
                \end{align*}
            \end{enumerate}

            \item[3.] Um móvel de massa $ 40 kg $ tem velocidade constante de $ 90 km/h $.
            Num determinado instante entra numa região rugosa onde o coeficiente de atrito é igual a $ 0,2 $. Determine:
            \begin{enumerate}
                \item[a)] o espaço percorrido pelo móvel na região rugosa, até parar;
                \begin{align*}
                             F_{at} &= 0,2 * 40 * 9,8             \\
                             F_{at} &= 78,4 newtons               \\
                               78,4 &= 40 * a                     \\
                    \frac{78,4}{40} &= a                          \\
                         1,96 m/s^2 &= a                          \\
                                0^2 &= 25^2 + 2 * 1,96 * \Delta S \\
                           \Delta S &= 159,43877551 metros
                \end{align*}
                
                \item[b)] o trabalho realizado pela força de atrito.
                \begin{align*}
                    \tau &= 78,4 * 159,43877551 * \cos(180) \\
                    \tau &= 12499,999999984 * -1            \\
                    \tau &= -12499,999999984 joules
                \end{align*}
            \end{enumerate}
        \end{enumerate}
    \end{day}
    
    \begin{day}{26/04/2021}
        \title{3}{Termometria}
        
        Temperatura é uma grandeza física que mede o estado de \textbf{agitação das partíulas} de um corpo, caracterizando o seu estado térmico.
        
        \title{3}{Calor}
        
        É energia térmica \textbf{em trânsito}, entre dois corpos ou sistemas, decorrente apenas da existência de uma diferença de temperatura entre eles.
        
        \title{3}{Escalas Termométricas}
        
        Uma escala termométrica corresponde a um conjunto de valores numéricos, onde cada um desses valores está associado a uma temperatura.
        
        Para a graduação das escalas foram escolhidos, para pontos fixos, dois fenômenos que se reproduzem sempre nas mesmas condições: a fusão do gelo e a ebulição da água, ambas sob pressão normal.
        
        \vspace{3pt}
        \begin{itemize}[nosep]
            \item Primeiro ponto fixo: corresponde à temperatura de fusão do gelo; chamado ponto do gelo;
            \item Segundo ponto fixo: corresponde à temperatura de ebulição da água; chamado ponto de vapor.
        \end{itemize}
        
        \title{3}{Relações entre as escalas}
        
        Supondo que a grandeza termométrica seja a mesma, podemos relacionar as temperaturas assinaladas pelas escalas termométricas da seguinte forma:
        \vspace{6pt}
        
        \begin{tikzpicture}\setmainfont{Latin Modern Roman}
            % Change from white to gray to enable grid
            \draw [very thin, white,step=1] (-8,0) grid (8,-5);
          % \filldraw [fill=black] (0,0) circle [radius=1.5pt];
            
            \coordinate (HTL) at (-3.5,0);
            \coordinate (HTR) at (3.5,0);
            
            \coordinate (HCL) at (-3.5,-2);
            \coordinate (HCR) at (3.5,-2);
            
            \coordinate (HBL) at (-3.5,-5);
            \coordinate (HBR) at (3.5,-5);
            
            \draw (HTL) -- (HTR);
            \draw (HCL) -- (HCR);
            \draw (HBL) -- (HBR);
            
            \coordinate (VLT) at (-2,0);
            \coordinate (VLM) at (-2,-2);
            \coordinate (VLB) at (-2,-5);
            
            \coordinate (VCT) at (0,0);
            \coordinate (VCM) at (0,-2);
            \coordinate (VCB) at (0,-5);
            
            \coordinate (VRT) at (2,0);
            \coordinate (VRM) at (2,-2);
            \coordinate (VRB) at (2,-5);
            
            \draw [|-|,thick] (VLT) -- (VLB);
            \draw [|-|,thick] (VCT) -- (VCB);
            \draw [|-|,thick] (VRT) -- (VRB);
            
            \draw [dashed] (VLT) to[out=0,in=0,distance=-1cm] (VLB);
            \draw [dashed] (VCT) to[out=0,in=0,distance=-1cm] (VCB);
            \draw [dashed] (VRT) to[out=0,in=0,distance=-1cm] (VRB);
            
            \draw [dashed] (VLM) to[out=0,in=0,distance=0.8cm] (VLB);
            \draw [dashed] (VCM) to[out=0,in=0,distance=0.8cm] (VCB);
            \draw [dashed] (VRM) to[out=0,in=0,distance=0.8cm] (VRB);
            
            \node [above,align=center] at (VLT) {Celsius   \\100};
            \node [above,align=center] at (VCT) {Kelvin    \\373};
            \node [above,align=center] at (VRT) {Fahrenheit\\212};
            
            \node [below,align=center] at (VLB) {0  };
            \node [below,align=center] at (VCB) {273};
            \node [below,align=center] at (VRB) {32 };
            
            \node [below left,align=center] at (VLM) {C};
            \node [below left,align=center] at (VCM) {K};
            \node [below left,align=center] at (VRM) {F};
            
            \node [align=right,left] at (HCL) {Temperatura\\qualquer};
        \end{tikzpicture}
        
        $$ \frac{C-0}{100-0} = \frac{K-273}{373-273} = \frac{F-32}{212-32} \Longrightarrow \frac{C}{100} = \frac{K-273}{100} = \frac{F-32}{180} $$
    \end{day}
    
    \begin{day}{27/04/2021}
        \title{3}{ATIVIDADE}
        
        \begin{enumerate}
            \item[1.] A temperatura normal de um corpo humano é 36°C. Qual é essa temperatura expressa nas escalas Fahrenheit e Kelvin?
            \begin{multicols}{2}
                \begin{align*}
                    \frac{36}{\frac{100}{20}} &= \frac{F - 32}{\frac{180}{20}} \\
                                 \frac{36}{5} &= \frac{F - 32}{9}              \\
                             \frac{F - 32}{9} &= \frac{36}{5}                  \\
                             \frac{F - 32}{9} &= 7,2                           \\
                                       F - 32 &= 7,2 * 9                       \\
                                       F - 32 &= 64,8                          \\
                                            F &= 64,8 + 32                     \\
                                            F &= 96,8 ^\circ Fahrenheit
                \end{align*}\\
                \begin{align*}
                    \frac{36}{\frac{100}{20}} &= \frac{K - 273}{\frac{100}{20}} \\
                                 \frac{36}{5} &= \frac{K - 273}{5}              \\
                            \frac{K - 273}{5} &= \frac{36}{5}                   \\
                            \frac{K - 273}{5} &= 7,2                            \\
                                      K - 273 &= 7,2 * 5                        \\
                                            K &= 36 + 273                       \\
                                            K &= 309 Kelvin
                \end{align*}
            \end{multicols}
            
            \item[2.] Transformar 104°F nas escalas Celsius e Kelvin.
            \begin{multicols}{2}
                \begin{align*}
                    \frac{104 - 32}{\frac{180}{20}} &= \frac{C}{\frac{100}{20}} \\
                                 \frac{104 - 32}{9} &= \frac{C}{5}              \\
                                       \frac{72}{9} &= \frac{C}{5}              \\
                                                  8 &= \frac{C}{5}              \\
                                        \frac{C}{5} &= 8                        \\
                                                  C &= 8 * 5                    \\
                                                  C &= 40 ^\circ Celsius
                \end{align*}
                \begin{align*}
                    \frac{104 - 32}{\frac{180}{20}} &= \frac{K - 273}{\frac{100}{20}} \\
                                 \frac{104 - 32}{9} &= \frac{K - 273}{5}              \\
                                       \frac{72}{9} &= \frac{K - 273}{5}              \\
                                                  8 &= \frac{K - 273}{5}              \\
                                  \frac{K - 273}{5} &= 8                              \\
                                            K - 273 &= 8 * 5                          \\
                                            K - 273 &= 40                             \\
                                                  K &= 40 + 273                       \\
                                                  K &= 313 Kelvin
                \end{align*}
            \end{multicols}
            
            \item[3.] Numa das regiões mais frias do mundo, o termômetro indica -76°F. Qual será o valor dessa temperatura na escala Celsius?
            \begin{align*}
                \frac{-76 - 32}{\frac{180}{20}} &= \frac{C}{\frac{100}{20}} \\
                             \frac{-76 - 32}{9} &= \frac{C}{5}              \\
                                 \frac{-108}{9} &= \frac{C}{5}              \\
                                            -12 &= \frac{C}{5}              \\
                                    \frac{C}{5} &= -12                      \\
                                              C &= -12 * 5                  \\
                                              C &= -60 ^\circ Celsius
            \end{align*}
            
            \item[4.] Ao medir a temperatura de um gás, verificou-se que a leitura era a mesma, tanto na escala Celsius como na Fahrenheit. Qual era essa temperatura?
            \begin{align*}
                \frac{T - 32}{\frac{180}{20}} &= \frac{T}{\frac{100}{20}} \\
                             \frac{T - 32}{9} &= \frac{T}{5}              \\
                                 (T - 32) * 5 &= 9T                       \\
                                     5T - 160 &= 9T                       \\
                                      5T - 9T &= 160                      \\
                                          -4T &= 160                      \\
                                            T &= \frac{160}{-4}           \\
                                            T &= -40^\circ
            \end{align*}
        \end{enumerate}
    \end{day}
\end{document}

