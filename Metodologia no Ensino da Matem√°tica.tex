\documentclass{SchoolBook}

\begin{document}
    \begin{day}{17/05/2021}
        \begin{enumerate}
            \item[1.] Qual o atual papel da educação matemática?
            \response{"Formar cidadãos aptos para o convívio em sociedade, respeitando as diferenças, agindo de forma crítica e reflexiva diante das situações cotidianas."}
            
            \item[2.] O que o aluno passa a entender?
            \response{O aluno entende que a matemática não é útil apenas dentro do ambiente escolar, mas fora dela, na sociedade.}
            
            \item[3.] Como a introdução da modelagem matemática pode ser feita?
            \response{A introdução pode ser feita através da resolução de problemas, trazendo para dentro de sala a realidade do aluno.}
            
            \item[4.] Quais os tipos de modelagem matemática?
            \response{Não há tipos predefinidos de modelagem matemática. Modelagem matemática é um padrão matemático criado para qualquer área do conhecimento.}
            
            \item[5.] Como a modelagem pode nos ajudar?
            \response{A criação de uma modelagem matemática ajuda na resolução dos problemas propostos criando fórmulas ou padrões matemáticos.}
        \end{enumerate}
    \end{day}
    
    \begin{day}{11/06/2021}
        \title{3}{Texto \emph{\href{https://drive.google.com/file/d/167RWN6Z-7H5BkAholpEdeZX116oSQB6u/view}{O ENSINO DA MATEMÁTICA ATRAVÉS DO LÚDICO NA EDUCAÇÃO INFANTIL}}. Comentar sobre afirmações}
        
        2 A Importância do Lúdico na Matemática na Educação Infantil
        
        Um assunto muito discutido ultimamente são as atividades lúdicas -- atividades que visam o divertimento -- e suas finalidades na Educação Infantil. Levar a criança a um conhecimento mais abstrato, misturar esforço com brincadeira, transformaria o aprendizado num jogo bem-sucedido.
        
        Atividades que visam o divertimento, o lúdico, proporcionam uma sensação de bem estar. Kishimoto afirma que o jogo é importante para o desenvolvimento infantil, pois proporciona a descontração, a adoção de regras, o trabalho do imaginário e a aquisição do conhecimento.
        
        Para Smole, ''o jogo é uma atividade séria que não tem consequência frustante para a criança.``
        
        O aprendizado lúdico desenvolve a confiança, fazendo com que a criança participe ativamente de cada atividade sem ter medo de errar.
        
        Assim é criado um ambiente cooperativo, em que as crianças aprendem a compartilhar, dividir e ajudar o próximo em qualquer situação.
        
        Para Kishimoto, o jogo estimula a exploração e a solução de problemas e, por ser livre de pressões, cria um clima adequado para a investigação e a busca de soluções.
        
        O jogo pode ser trabalhado de várias formas ou/e combinações diferentes, mas deve criar um espaço de confiança e criatividae para ser desenvolvido de maneira lúdica.
        
        Por isso, como Smole cita, o jogo, propiciando situações problema, exige soluções criativas e rápidas. Assim a criança adquire conhecinento das tentativas anteriores ou planejamento.
        
        Com os jogos a criança aprende a desenvolver estratégias, essas necessárias para a conclusão do mesmo.
        
        2.1 O Brincar
        
        A brincadeira simula situações reais de forma simplificada. Com a brincadeira a criança finge estar num outro papel, finge ser ou se colocar no lugar de outra pessoa.
        
        A imaginação é um processo muito novo para a criança, e é com jogos que a criança desenvolve a mesma.
        
        Para aqueles fora do ambiente de educação, o brincar é apenas uma passa-tempo para a criança, deixando ela fazer o que quiser. Mas o brincar tem o objetivo de desenvolver a aprendizagem.
        
        O brincar também permite do professor compreender melhor a criança, como descobrir a atividade mais prezerosa para ela.
        
        2.2
        
        A criança aprende, em sua maioria, brincando com jogos. Nesses ela aprende conceitos como regras, tempo, colaboração e imaginação. Por isso é tão importante a criança ter contato com jogos quando criança. Ela aprende os conceitos sem se dar conta que está aprendendo, é algo natural.
        
        2.3 O Jogo e a Matemática
        
        Não é de hoje que os alunos estão menos interessados em matemática. Esse é um problema já bem velho. Por esse motivo os jogos foram trazidos para a sala de aula tornando o aprendizado mais lúdico e comum.
        
        Kishimoto diz que, dessa perspectiva, o jogo tem a finalidade de desenvolver habilidades de resolução de problemas.
        
        Os jogos proporcionam a sensação de prazer e bem estar, os mesmos, junto com a matemática, desenvolvem gosto pelos números e um pensamento matemático prático.
        
        Essa junção de jogo com matemática se tornou mais do que necessário, pois é brincando que as crianças melhor aprendem.
        
        Os professores devem deixar bem claro as regras e começar o jogo apenas quando todos os alunos tenham as compreendido.
        
        Trabalhar o significado das regras nos jogos faz com que a criança cresça entendendo que para ela conseguir o que quer precisa seguir regras, essas que muitas vezes as impedem de irem pelo caminho mais fácil.
        
        3 O Ensino de Matemática de Acordo com Referencial Curricular Nacional de Educação Infantil
        
        A criança tem a capacidade de absorver qualquer conhecimento, esses que serão levados ao longo da vida.
        
        A escola utiliza das atividades do dia a dia como ponto de partida para a criança adquirir o conhecimento, dando continuidade aumentando o mesmo. É nesse período que a criança desenvolverá suas bases, senso esse o melor momento para o aprendizado da matemática.
        
        Ao falarmos em matemática já pensamos em numeros, cálculos, quantidades, mas ela abrange muito mais do que isso.
        
        Sempre estamos utilizando a matemática de uma forma mais natural no mundo. Deis de dividir doces até calcular se vale a pena compra algo.
        
        Sendo assim, quanto mais cedo forem trabalhados os conceitos matemáticos melhor será os resultados no futuro, quando os alunos forem para o Ensino Fundamental ou Médio.
        
        É por causa de todos os benefícios que o aprendizado de matemática nos proporciona que o método utilizado pelos professores vem sendo discutido.
        
        3.1 Números e Sistema de Numeração
        
        É esperado que a criança aprenda a lidar com numeros, contagem e poderem resolver problemas matemáticos básicos.
        
        Para isso, pode ser trabalhado contagem oral, como jogos, brincadeiras ou músicas que explorem a quantificação.
        
        3.2 Grandezas e Medidas
        
        Medidas são uma boa forma de compreender os números, principalmente referenciais. Essas estão presentes o tempo todo no dia a dia das crianças, como ao comparar qual brinquedo é mais pesado ou maior, ou qual está mais perto ou distante.
        
        Existem muitas maneiras criativas de trabalhar com esses conceitos. Pode ser trabalhado alimentos quentes ou frios -- desenvolvendo senso de temperatura --, medida do ambiente escolar ou amigos. Também pode ser trabalhado senso de tempo utilizando calendários, datas comemorativas, etc.
        
        3.3 Espaço e Forma
        
         Esses conceitos propiciam que a criança aprenda a identificar objetos e figuras, tipos de contornos, pontos de referência, etc.
         
         A modelagem manual é uma atividade que explora muito bem os mesmos, pois a criança pode representar a realidade utilizando diferentes materiais, como massa de modelar, areia, argila, ou mesmo desenhando.
         
         Construção de maquetes também é uma atividade importante, permitindo a exploração do espaço, tamanho, pontos de vista, formas, dimensões, organização, referências mentais, etc.
         
         É através da curiosidade que as crianças exploram o mundo, descobrindo cada vez mais, criando conceitos através de atividades relacionando-as com a realidade.
         
         O professor sempre deve ter em mente que: explorar o conhecimento prévio do aluno, atribuindo conhecimento baseado em seus gostos, é a maneira mais adequada de o estimular a procurar mais sobre aquilo.
         
         É fundamental que o professor veja o aluno como ele é: uma criança. Portanto, é preciso se envolver no seu mundo, abusar da imaginação e fantasias e desenvolver atividades lúdicas tornando o aprendizado natural como o brincar.
    \end{day}
    
    \begin{day}{23/07/2021}
        \title 3 {Texto \emph{\href{https://drive.google.com/file/d/1FeqgliGuIMdNtf25xBYJ4UcUIq3gW_OJ}{Grandezas e medidas, Números, Álgebra, Geometria, Probabilidade e estatística: entenda o que é esperado para cada um desses eixos}}. Comentar sobre afirmações.}
        
        Álgebra
        
        A Álgebra, utilizando do pensamento algébrico, permite o ser compreender e representar relações de grandezas, equivalências, variação, interdependência ou proporcionalidade. Esse conteúdo prepara o aluno para: perceber padrões (numétricos ou não); interpretar representações gráficas e simbólicas; e resolver problemas por meio de equações ou inequações.
        
        Números
        
        Essa unidade tem como objetivos desenvolver o pensamento numérico -- sendo esse relacionado à capacidade de contar, quantificar, julgar e interpretar quantidades -- e as noções de aproximação, proporcionalidade equivalência e ordem.
        
        Geometria
        
        Essa unidade tem como objetivo estudar o deslocamento, relações ou transformações em espaços, formas ou figuras geométricas.
        
        Grandezas e Medidas
        
        O conhecimento em grandezas e medidas permite fazer relações ou intercomunicar os outros campos, também ajudando na compreensão dos mesmos.
        
        Probabilidade e Estatística
        
    \end{day}
    
    \begin{day}{...}
        
    
        Fazer números do 0 até o 9 de: tamanhos variados, cores variadas (claras e escuras), texturas variadas (áspero e liso), bordas variadas (arredondado e pontudo) e expessuras variadas (fino e grosso).
        (Todos esses atributos em uma unica sequência)
        
        Numeros: 0 1 2 3 4 5 6 7 8 9
        Sendo:
            5 claros      , 5 escuros
            3 ásperos     , 7 lisos
            6 arredondados, 4 pontudos
            8 finos       , 2 grossos
            
            0: 15,0cm, claro , liso  , arredondado, fino  ;
            1: 14,5cm, escuro, áspero, pontudo    , fino  ;
            2: 14,0cm, claro , liso  , pontudo    , grosso;
            3: 13,5cm, escuro, liso  , arredondado, fino  ;
            4: 13,0cm, claro , áspero, pontudo    , fino  ;
            5: 12,5cm, escuro, liso  , arredondado, fino  ;
            6: 12,0cm, claro , liso  , arredondado, fino  ;
            7: 11,5cm, escuro, áspero, pontudo    , grosso;
            8: 11,0cm, claro , liso  , arredondado, fino  ;
            9: 10,5cm, escuro, liso  , arredondado, fino  ;
            
        Fazer os alunos ordenarem os numeros em ordem crescente numérica
        Fazer os alunos ordenarem os numeros em ordem crescente de tamanho
        
        Fazer os alunos separarem os de borda arredondada
        Fazer os alunos separarem os de borda pontuda
        Contar quantos ficaram de cada
        
        Fazer os alunos separarem os com textura áspera
        Fazer os alunos separarem os com textura lisa
        Contar quantos ficaram de cada
        
        Fazer os alunos separarem os com cor escura
        Fazer os alunos separarem os com cor clara
        Contar quantos ficaram de cada
        
        Fazer os alunos separarem os com espessura grossa
        Fazer os alunos separarem os com espessura fina
        Contar quantos ficaram de cada
        
    \end{day}
\end{document}
