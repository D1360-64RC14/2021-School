\documentclass{SchoolBook}

\begin{document}
    \begin{day}{17/05/2021}
        \begin{enumerate}
            \item[1.] Qual o atual papel da educação matemática?
            \response{"Formar cidadãos aptos para o convívio em sociedade, respeitando as diferenças, agindo de forma crítica e reflexiva diante das situações cotidianas."}
            
            \item[2.] O que o aluno passa a entender?
            \response{O aluno entende que a matemática não é útil apenas dentro do ambiente escolar, mas fora dela, na sociedade.}
            
            \item[3.] Como a introdução da modelagem matemática pode ser feita?
            \response{A introdução pode ser feita através da resolução de problemas, trazendo para dentro de sala a realidade do aluno.}
            
            \item[4.] Quais os tipos de modelagem matemática?
            \response{Não há tipos predefinidos de modelagem matemática. Modelagem matemática é um padrão matemático criado para qualquer área do conhecimento.}
            
            \item[5.] Como a modelagem pode nos ajudar?
            \response{A criação de uma modelagem matemática ajuda na resolução dos problemas propostos criando fórmulas ou padrões matemáticos.}
        \end{enumerate}
    \end{day}
    
    \begin{day}{11/06/2021}
        \title{3}{Texto \href{https://drive.google.com/file/d/167RWN6Z-7H5BkAholpEdeZX116oSQB6u/view}{O ENSINO DA MATEMÁTICA ATRAVÉS DO LÚDICO NA EDUCAÇÃO INFANTIL}. Comentar sobre afirmações}
        
        
    \end{day}
\end{document}
