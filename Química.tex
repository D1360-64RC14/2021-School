\documentclass{SchoolBook}

\usepackage{graphicx}
\usepackage{chemfig}

\begin{document}
    \begin{day}{22/04/2021}
        \title{3}{Grupos de Hidocarbonetos de Cadeia Fechada}
        Ciclanos: São hidrocarbonetos de cadeia fechada com ligação simples antre átomos de carbono.
    \end{day}
    \begin{day}{29/04/2021}
        \title{3}{Grupos de Hidrocarbonetos de Cadeia Fechada}
        
        \textbf{Ciclenos}: são hidrocarbonetos de cadeia cíclica ou fechada com ligação dupla entre os átomos de carbono. Exemplos:
        
        \begin{center}
            \begin{tabular}{c c}
                \chemfig{H -[:53] C ** 5 (=[:0]C(-H) - C(-[:-45]H)(-[:45]H) - C(-[:45]H)(-[:135]H) - C(-[:225]H)(-[:135]H) - C(-H)(-H))} &
                \chemfig{[:18]() * 5 (= () - () - () - () -)} \\[12pt]
                \chemfig{H_2C ** 4(-C(-[:0]H) - C(-[:0]H) - H_2C -[,,2])} &
                \chemfig{() * 4 (- () = () - () - () -)} \\[12pt]
                \chemfig{H -[:60] C ** 6 (= C(- H) - CH_2 - C(- H_2) - C(- H_2) - H_2C -)} &
                \chemfig{[:30]() * 6 (= () - () - () - () - () -)}
            \end{tabular}
        \end{center}
        
        \title{3}{Nomeclaturas Usuais de Alguns Aromáticos}
        
        \title{3}{Atividade}
        
        Forneça a nomeclatura segundo a IUPAC para o composto a seguir:
        
        \vspace{6pt}
        \begin{tabular}{|l|l|}
            \chemfig{[:90]() * 6 (- () = () - () = ()(- CH_3) - () =)} & Metilbenzeno \\[6pt]
            \chemfig{()(-[:180]CH_3) * 6 (= () - () = ()(-[:0] CH_3) - () = () -)} & 1,4-dimetilbenzeno
        \end{tabular}
        
        \title{2}{Representação Estrutural dos Hidrocarbonetos e Aplicabilidade}
        
        \title{3}{Fórmula Estrutural Plana}
        
        É a representação que mostra todos os átomos e todas as ligações presentes na molécula, pois apresenta a disposição ou arranjo dos átomos dentro de uma molécula. Exemplo:
        
        \begin{center}
            \chemfig{
                  H
                - C(-[:90] H) (-[:-90] H)
                - C(-[:90] C(-[:45] H) (-[:90] H) (-[:135] H)) (-[:-90] C(-[:-45] H) (-[:-90] H) (-[:-135] H))
                - C(-[:90] H)(-[:-90] H)
                - C(-[:90] H)(-[:-90] C(-[:-45] H)(-[:-90] H) (-[:-135] H))
                - C(-[:90] H) (-[:-90] H)
                - H
            }
        \end{center}
        
        Para esta representação, a quantidade de hidrogênios é abreviada. Exemplo:
        
        \begin{center}
            \chemfig{H_3C - C(-[:90] CH_3)(-[:-90] CH_3) - CH_2 - CH(-[:-90] CH_3) - CH_3}
        \end{center}
        
        Trata-se da fórmula que unifica o número de átomos de cada elemento em uma molécula das substâncias.
        
        \begin{center}
            \chemfig{H_3C - C(-[:90]CH_3)(-[:-90]CH_3) - CH_2 - CH(-[:-90]CH_3) - CH_3}
            \hspace{12pt}
            \chemfig{C_{8}H_{18}} --- Fórmula Molecular correspondente
        \end{center}
        
        Fórmula estrutural plana simplificada ou condensada
        
        Assim, para determinar a fórmula molecular desse composto, basta contar a quantidade de átomos de cada elemento e colocar um índice do lado inferior direito do elemento em questão. Exemplos de Fórmula estrutural simplificada ou condensada e fórmula molecular, respectivamente:
        
        \begin{center}
            \begin{tabular}{r c}
                \chemfig{CH_3 - CH_3}             & \chemfig{C_{2}H_{6}} \\
                \chemfig{CH_3 - CH_2 - CH_3}      & \chemfig{C_{3}H_{8}} \\
                \chemfig{CH_2 = CH - CH_2 - CH_3} & \chemfig{C_{4}H_{8}}
            \end{tabular}
        \end{center}
    \end{day}
    
    \begin{day}{13/05/2021}
        \title{3}{Representação dos Hidrocarbonetos}
        
        Uma das primeiras observações na Química Orgânica são as diferentes maneiras que podemos representar uma mesma molécula.
        
        \begin{center}
            \begin{tabular}{c l}
                \chemfig{() -[:30] () -[:-30] () -[:30] () -[:-30] ()} & Bastão \\[24pt]
                \chemfig{C(-[:180] H)(-[:90] H)(-[:-90] H) - C(-[:90] H)(-[:-90] H) - C(-[:90] H)(-[:-90] H) - C(-[:90] H)(-[:-90] H) - C(- H)(-[:90] H)(-[:-90] H)} & Estrutural Plana \\[50pt]
                \chemfig{H_3C - CH_2 - CH_2 - CH_2 - CH_3} & Estrutural Condensada/Simplificada \\[24pt]
                \chemfig{C_5H_{12}} & Fórmula Molecular
            \end{tabular}
        \end{center}
        
        \title{3}{Fórmula em Perspectiva}
        
        \begin{center}
            \begin{tabular}{c l}
                \chemfig{H - H} & No Plano \\
                \chemfig{H > H} & Para Trás \\
                \chemfig{H >: H} & Para Frente
            \end{tabular}
        \end{center}
        
        \title{3}{Exemplo}
        \begin{center}
            \chemfig{OH -[:30] () -[:-30] () (<[:-150] H) (<:[:-80] OH) -[:30] () (<[:120] H) (<:[:60] OH) -[:-20] () =[:20] O}
        \end{center}
        
        \title{3}{Atividade}
        
        A substância responsável pelo sabor característico da laranja apresenta a seguinte estrutura simplificada:
        
        \chemfig[atom sep=2em]{O(-[:90] C(-[:-180] C) =[:90] O) - C - C - C - C - C - C - C - C}
        
        \begin{enumerate}
            \item[a)] Qual é o número de átomos de hidrogênio presentes em uma molécula dessa substância?
            \response{20 hidrogênios}
            
            \item[b)] Qual a fórmula molecular dessa substância?
            \response{\chemfig{H_{20}C_9O_2}}
            
            \item[c)] Essa molécula possui algum heteroátomo? 
            \response{Sim.}
        \end{enumerate}
        
        A metformina é um antidiabético que pertence à classe de remédios chamados de biguanidas -- agentes utilizados como hipoglicemiantes -- ou seja, eles são capazes de reduzir/controlar a concentração de glicose na corrente sanguínea. A fórmula estrutural deste fármaco está representada a seguir.
        
        \vspace{12pt}
        \heading{2}{\chemfig[atom sep=2em, bond style={line width=1pt},bond join=true]
        {H_3C -[:30] N(-[:-90] CH_3) -[:-30] ()(-[:90] NH_2) =^[:30] N -[:-30] ()(=[:90] NH) -[:30] NH_2}}
        \vspace{12pt}
        
        \noindent Determine a fórmula molecular.
        \response{\chemfig{H_{11}C_4N_5}}
    \end{day}
    
    \begin{day}{20/05/2021}
        A bioluminescência é o fenômeno responsável pela emissão de luz dos vaga-lumes. Tal luz é produzida em razão da oxidação de uma substância chamada luciferina, cuja fórmula está representada a seguir. Determine a fórmula molecular.
        
        \begin{center}
            \chemfig[atom sep=2em]{
                HO - () * 6(= () - () = () * 5(- N = ()(- () * 5(= N - ()(<: ()(=[::-45] O)(- OH)) - () - S -)) - S - () -) - () = () -)
            }
        \end{center}
        
        \title{3}{Propriedades Físicas e Químicas dos Hidrocarbonetos}
        
        \title{4}{Compostos Orgânicos}
        
        Os compostos otgânicos são aqueles que possuem como elemento principal o caborno.
        
        \begin{center}
            \heading{4}{\bf H, O, N, halogênios (flúor, cloro, bromo e iodo) e S.}
        \end{center}
        
        \title{3}{Ponto de Fusão e Ebulição}
        
        Apresentam, em sua grande maioria, \textbf{pontos de fusão e ebulição baixos:}
        
        \vspace{6pt}
        \begin{tabular}{|c|c|c|c|}
            \hline
                             & \chemfig{CH_4} & \chemfig{H_3C-CH_3} & \chemfig{H_3C-CH_2-CH_3}\\
            \hline
            \textbf{TE (°C)} & -188 & -88,4 & -42,5\\
            \hline
                             & \chemfig{H-C(=[::20]O)(-[::-20]H)} & \chemfig{G_3C-C(=[::20]O)(-[::-20]H)} & \chemfig{H_3C-CH2-C(=[::20]O)(-[::-20]H)} \\
            \hline
            \textbf{TE (°C)} & -19 & 20 & 48,8 \\
            \hline
                             & \chemfig{H_3C-OH} & \chemfig{H_3C-CH_2-OH} & \chemfig{H_3C-CH_2-CH_2-OH}\\
            \hline
            \textbf{TE (°C)} & 64,5 & 78,3 & 97,2\\
            \hline
        \end{tabular}
        \vspace{6pt}
        
        Quanto mais carbonos, maior é a temperatura de ebulição.
        
        $$\mathrm{CH_3(CH_2)_3CH_3 \longrightarrow CH_3 - CH_2 - CH_2 - CH_2 - CH_3}$$
    \end{day}
    
    \begin{day}{2º Trimestre --- 27/05/2021}
        \title{3}{Funções Químicas Oxigenadas}
        
        São compostos orgânicos que possuem, além de carbono e hidrogênio, átomos de oxigênio.
        
        \title{1}{Álcool}
        \title{3}{Características, nomeclatura e aplicações}
        
        Apresenta hidroxila \chemfig[atom sep=20pt]{-OH} ligada a carbono saturado. \textit{(carbono saturado é aquele que faz apenas ligações simples)}
        
        \begin{center}
            \chemfig[atom sep=24pt]{-C(-[:90])(-[:-90])-OH}
            \vspace{12pt}\par
            \begin{tabular}{c c c}
                \textbf{Mono álcool} & \textbf{Di álcool} & \textbf{Tri álcool} \\
                Apresenta uma hidroxila & apresenta duas hidroxilas & apresenta três hidroxilas \\
                \chemfig{CH_3-OH} & \chemfig{CH_2(-[:-90]OH)-CH_2(-[:-90]OH)} & \chemfig{CH_2(-[:-90]OH)-CH(-[:-90]OH)-CH_2(-[:-90]OH)}
            \end{tabular}
            \vspace{12pt}\par
            \begin{tabular}{c c c}
                \textbf{Álcool Primário} & \textbf{Álcool Secundário} & \textbf{Álcool Terciário} \\
                hidroxila em carbono primário & hidroxila em carbono secundário & hidroxila em carbono terciário \\
                \chemfig{H_3C-C(-[:90]O\colorx{red}{H})(-[:-90]H)-\colorx{red}{H}} & \chemfig{H_3C-C(-[:90]OH)(-[:-90]H)-H_3C} & \chemfig{H_3C-C(-[:90]OH)(-[:-90]H)-CH_3}
            \end{tabular}
            
            \title{3}{Nomeclatura de Álcoois}
            
            \begin{tabular}{|c|c|c|}
                \hline
                \textbf{PREFIXO} & \textbf{INTERMEDIÁRIO} & \textbf{SUFIXO} \\
                \hline
                \colorx{red}{Número de Carbonos} & \colorx{green}{Ligação} & \colorx{blue}{Função -- OL} \\
                \hline
            \end{tabular}
            
            \vspace{6pt}
            \chemfig{H_3C - CH_2 - OH} \\
            \textbf{\colorx{red}{ET} \colorx{green}{AN} \colorx{blue}{OL}}
            \vspace{6pt}
            
            \fbox{\parbox{13cm}{
                A nomeclatura seque as regras já vistas para os hidrocarbonetos. \\
                O sufixo OL e o carbono com a hidroxila (OH) \underline{deve fazer parte} da cadeia principal.
            }}
        \end{center}
        
        \title{3}{Atividade}
        
        \begin{enumerate}
        \item[1.] Você sabia da existência de tantas concentrações diferentes para o álcool etílico?
        \response{Sim.}
        
        \item[2.] Desenhe em seu caderno a estrutura desta molécula, lembrando que seu nome oficial é ETANOL. \\
        \chemfig{CH_3 - CH_2 - OH}
        \end{enumerate}
        
        \title{3}{Função Química -- ENOL}
        
        Apresentam um ou mais readicais hidroxilas (-- OH) ligado(s) à átomos(s) de carbono(s) insaturado(s).
        \vspace{12pt}
        
        \begin{center}
            Fórmula geral do composto: \\[6pt]
            \chemfig{C(-[:135]R^1)(-[:-135]R^2) = C(-[:45]OH)(-[:-45]R^3)}
        \end{center}
        
        \vspace{12pt}
        Já aprendemos que a função álcool apresenta (OH) ligado à um carbono saturado.
        
        Agora apresentamos o ENOL com uma \underline{dupla ligação entre carbonos}. Assim, percebemos que o \textbf{EN} refere-se à dupla ligação \textbf{OL} ao grupamento OH.
    \end{day}
    
    \begin{day}{10/06/2021}
        \title{2}{Diferenciando Álcoois e Enóis}
        
        \begin{tabular}{rcl}
            \chemfig{CH_2 = CH - CH_2} & $ \longrightarrow $ & Álcool insaturado OH em carbono saturado. \\
            \chemfig{CH_3 - CH = CH - OH} & $ \longrightarrow $ & Enol OH em carbono com ligação dupla.
        \end{tabular}
        
        \title{3}{Nomeclatura de Enóis}
        
        \begin{center}
            \begin{tabular}{|c|c|c|}
                \hline
                Prefixo & Intermediário & Sufixo \\
                \hline
                \colorx{red}{Numero de Carbonos} &
                    \colorx{green}{Ligação} &
                    \colorx{blue}{Função -- OL} \\
                \hline
            \end{tabular}
            \vspace{12pt}
            
            \chemfig{H_2C = CH - OH}
            \vspace{8pt}
            
            \colorx{red}{ET} \colorx{green}{EN} \colorx{blue}{OL}
        \end{center}
        
        \title{2}{Fenol -- Características, Nomeclatura e Aplicações}
        
        Os fenóis são compostos que apresentam o grupo hidroxila (-- OH) ligado diretamente a um átomo do anel aromático.
        
        \begin{center}
            \chemfig{C_{aromático} - OH}
        \end{center}
        
        \title{3}{Álcool, Enol, Fenol?}
        
        \begin{center}
            \begin{tabular}{p{5cm} | p{5cm} | p{5cm}}
                \chemfig{OH -[:-90] *6(- - - - - -)} &
                \chemfig{OH -[:-90] *6(- - - - - =)} &
                \chemfig{OH -[:-90] *6(= - = - = -)} \\
                \textbf{Alcool}: -- OH ligado à um C\_saturado. &
                \textbf{Enol}: -- OH ligado a C\_insaturado por dupla ligação. &
                \textbf{Fenol}: -- OH ligado à C\_aromático.
            \end{tabular}
        \end{center}
        
        \title{3}{Nomeclatura do Fenol}
        
        Na nomeclatura oficial, o grupo (-- OH) é denominado hidroxi e vem seguido do nome do hidrocarboneto.
        
        \begin{center}
            \begin{tabular}{c|c}
                \begin{tabular}{c}
                    \chemfig{OH -[:-90] *6(= - = - = -)} \\
                    Oficial: Hidroxibenzeno
                \end{tabular} &
                
                \begin{tabular}{c}
                    \chemfig{OH -[:-90] **6(- - - - - -)} \\
                    Usual: Fenol
                \end{tabular} \\
                \hline
                \begin{tabular}{c}
                    \chemfig{OH -[:-90] *6(= - = - = ()(-CH_3) -)} \\
                    Oficial: 1-hidroxi-2-metil-benzeno \\
                    Usual: 2-metil-benzeno
                \end{tabular} &
                
                \begin{tabular}{c}
                    \chemfig{OH -[:-90] *6(= - = - ()(-CH_3) = -)} \\
                    Oficial: 1-hidroxi-3-metil-benzeno \\
                    Usual: 3-metil-benzeno
                \end{tabular} \\
                \hline
                \begin{tabular}{c}
                    \chemfig{OH -[:-90] *6(= - = ()(-CH_3) - = -)} \\
                    Oficial: 1-hidroxi-4-metil-benzeno \\
                    Usual: 4-metil-benzeno
                \end{tabular} &
            \end{tabular}
        \end{center}
    \end{day}
    
    \begin{day}{17/06/2021}
        \title 3 {Atividade -- Nomeclatura}
        
        \begin{center}
            \begin{tabular}{c c}
                \chemfig{OH - *6(= - = - = ()(- CH_2CH_3) -)} & \chemfig{OH - *6(= - = - ()(- CH_2CH_3) = -)} \\
                Oficial: 1-hidroxi-2-etil-benzeno & Oficial: 1-hidroxi-3-etil-benzeno \\
                Usual: 2-etil-fenol & Usual: 3-etil-fenol
            \end{tabular}\\
            \vspace{12pt}
            \noindent\chemfig{OH - *6(= - = ()(- CH_2CH_3) - = -)} \\
            Oficial: 1-hodrixi-4-etil-benzeno \\
            Usual: 4-etil-fenol
        \end{center}
        
        \title 2 {Funções Químicas Oxigenadas: \underline{Éter} -- Características, Nomeclatura e Aplicações}
        
        \title 3 {Nomeclatura -- 1ª Maneira}
        
        Prefixo que indica o número de carbonos no menor radical + OXI + nome do hidrocarboneto correspondente ao maior radical
        \vspace{6pt}
        
        \begin{center}
            \begin{tabular}{c|c|c}
                \chemfig{\colorx{green}{CH_3} - O - \colorx{orange}{CH_3}} &
                \chemfig{\colorx{orange}{CH_3 - CH_2} - O - \colorx{green}{CH_3}} &
                \chemfig{\colorx{green}{CH_3 - CH_2} - O - \colorx{orange}{CH_2 - CH_3}} \\
                \color{green}MET \color{black}OXI \color{orange}METANO &
                \color{green}MET \color{black}OXI \color{orange}ETANO &
                \color{green}ET \color{black}OXI \color{orange}ETANO \\
                METOXIMETANO &
                METOXIETANO &
                ETOXIETANO
            \end{tabular}
        \end{center}
        \vspace{12pt}
        
        As duas substâncias a seguir são exemplos de éteres e estão representados por meio da sua grafia em bastão. Sobre elas:
        \begin{enumerate}[nosep]
            \item[a)] Represente a sua fórmula estrutural simplificada;
            \item[b)] De acordo com as regras de nomeclatura vistas, dê o nome oficial à cada uma delas.
        \end{enumerate}
        \vspace{6pt}
        
        \begin{tabular}{|c|c|}\hline 
            \chemfig{-[:45] O -[:-45] -[:45] -[:-45]} &
            \begin{tabular}{c}
                \chemfig{CH_3 - O - CH_2 - CH_2 - CH_3} \\
                MET OXI PROP ANO
            \end{tabular}\\\hline
            \chemfig{-[:45] -[:-45] -[:45] O -[:-45] -[:45] -[:-45]} &
            \begin{tabular}{c}
                \chemfig{CH_3 - CH_2 - CH_2 - O - CH_2 - CH_2 - CH_3} \\
                PROP OXI PROP ANO
            \end{tabular} \\\hline
        \end{tabular}
        \vspace{6pt}
        
        \title 3 {Nomeclatura -- 2ª Maneira}
        
        \begin{center}
            \colorx{blue}{ÉTER} + \colorx{green}{nome do menor radical} + \colorx{orange}{nome do maior radical} + \colorx{blue}{ICO}
            
            \vspace{6pt}
            \begin{tabular}{c|c|c}
                \chemfig{CH_3 - O - CH_3} &
                \chemfig{CH_3 - CH_2 - O - CH_3} &
                \chemfig{CH_3 - CH_2 - O - CH_2 - CH_3} \\
                ÉTER METIL METIL ICO &
                ÉTER METIL ETIL ICO &
                ÉTER ETIL ETIL ICO \\
                \begin{tabular}{c}
                    ÉTER METIL-METÍLICO \\
                    ÉTER DIMETÍLICO \\
                    ÉTER METÍLICO
                \end{tabular} & &
                \begin{tabular}{c}
                    ÉTER ETIL-ETÍLICO \\
                    ÉTER DIETÍLICO \\
                    ÉTER ETÍLICO
                \end{tabular}
            \end{tabular}
        \end{center}
    \end{day}
    
    \begin{day}{24/06/2021}
        \begin{center}
            \chemfig{CH_3 - C(-[:90] CH_3)(-[:-90] CH_3) - O - CH_2 - CH_3}
            \vspace{6pt}
            
            \begin{tabular}{r c l}
                ET OXI METILPROPANO &$\rightarrow$& 2-ETOXI-2-METILPROPANO \\
                ÉTER ETIL TERCBUTIL ICO &$\rightarrow$& ÉTER ETIL-TERCBUTÍLICO
            \end{tabular}
            
            \vspace{24pt}
            \chemfig{CH_2 = CH - O - CH = CH_2}
            \vspace{6pt}
            
            \begin{tabular}{r c l}
                VINIL OXI ETENO &$\rightarrow$& ETENOXIETENO \\
                ÉTER VINIL VINIL ICO &$\rightarrow$& ÉTER VINÍLICO
            \end{tabular}
        \end{center}
    \end{day}
    
    \begin{day}{08/07/2021}
        \title 3 {Funções Químicas Oxigenadas:\\Aldeído -- Características, Nomeclatura e Aplicaçõeos}
        
        Os aldeídos apresentam o grupo carbonila (\chemfig{C=O}) na extremidade da cadeia.
        
        \chemfig{R-C(=O)-H}
        
        \textbf{R} é um radical orgânico. No aldeído mais somples, \textbf{R} é o hidrogênio (H).
        
        \begin{tabular}{|c|c|c|}
            \hline
            \textbf{Prefixo} & \textbf{Intermediário} & \textbf{Sufixo} \\
            \hline
            \colorx{red}{Número de Carbonos} & \colorx{green}{Ligação} & \colorx{blue}{Função -- AL} \\
            \hline
        \end{tabular}
        
        A nomeclatura segue as regras já vistas para os hidrocarbonetos. O Sufixo \textbf{AL} e o carbono com a carbonila (\chemfig{C=O}) \underline{deve estar na extremidade} da cadeia.
        
        \chemfig{R-C(=O)-H} \colorx{red}{MET}\colorx{green}{AN}\colorx{blue}{AL}
        
        \begin{tabular}{c|c|c}
                                   & \chemfig{C(-H)(<:H)(<H) - C(=O)(-H)}
                                   & \chemfig{C(-H_3C)(<:H)(<H) - C(=O)(-H)}          \\
            \hline
            Nº de Carbonos         & 2: \colorx{red}{ET}
                                   & 3: \colorx{red}{PROP}                            \\
            Ligação entre carbonos & \multicolumn{2}{c}{\colorx{green}{simples - AN}} \\
            Sufixo                 & \multicolumn{2}{c}{\colorx{blue}{AL} (Aldeído)}  \\
            \hline
        \end{tabular}
        
        Os quatro aldeídos mais simples apresentam nomes usuais formados pelos prefixos: form, acet, propion, butir, seguidos da palavra aldeído.
        
        \chemfig{H-C(=O)-H} METANAL - FORMALDEÍDO
        
        \chemfig{H-C(-H)(-H)-C(=O)-H} ETANAL - ACETALDEÍDO
        
        \chemfig{H-C(-H)(-H)-C(-H)(-H)-C(=O)-H} PROPANAL - PROPIONALDEÍDO
        
        \chemfig{H-C(-H)(-H)-C(-H)(-H)-C(-H)(-H)-C(=O)-H} BUTANAL - BUTIRALDEÍDO
    \end{day}
    
    \begin{day}{22/07/2021}
        \title 3 {Funções Químicas Oxigenadas:\\Cetona -- Características, Nomeclatura e Aplicaçõeos}
        
        As acetonas apresentam o grupo carbonila (\chemfig{C=O}), sendo este carbono secundário.
        
        \chemfig{()([:90]=O)([:225]-R)([:-45]-R_1)}
        
        Um carbono secundário é aquele que apresenta duas ligações diretamente com outros carbonos.
        
        \chemfig{\colorx{green}{C}([:90]=O)([:-45]-CH_2-CH_3)(-[:225]CH_2-CH_3)}
        
        ---------------------------------------------
        \begin{tabular}{|c|c|c|}
            \hline
            \textbf{Prefixo} & \textbf{Intermediário} & \textbf{Sufixo} \\
            \hline
            \colorx{red}{Número de Carbonos} & \colorx{green}{Ligação} & \colorx{blue}{Função -- ONA} \\
            \hline
        \end{tabular}
        
        A nomeclatura segue as regras já vistas para os hidrocarbonetos. O Sufixo \textbf{ONA} e o carbono com a carbonila (\chemfig{C=O}) \underline{deve ser um carbono secundário}.
        
        \chemfig{O[:90]=C([:45]-C(-H)(-H)(-H))([:135]-C(-H)(-H)(-H))}
        
        \colorx{red}{PROP}\colorx{green}{AN}\colorx{blue}{ONA}
        
        \title 3 {Atividade}
        
        Dê o nome aos seguintes compostos:
        
        1) \chemfig{O=C(-CH_2-CH_3)(-CH_2-CH_3)}
        
        2) \chemfig{O=C(-CH_2-CH_2-CH_3)(-CH_3)}
        
        \noindent 5 carbonos: PENT \\
        ligação: simples (AN)  \\
        sufixo: cetona (ONA)
        
        1) PENTAN-3-ONA \\
        2) PENTAN-2-ONA
        
        Existe uma nomeclatura usual em que o grupo \chemfig{C=O} é denominado \textbf{cetona}, e seus ligantes são considerados radicais.
        
        \noindent 1) ETIL-EIL-CETONA, ou DIETILCETONA, ou CETONA DIETÍLICA \\
        2) METIL-PROPIL-CETONA, ou CETONA METIL-PROPÍLICA
    \end{day}
\end{document}
