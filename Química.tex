\documentclass{SchoolBook}

\usepackage{graphicx}
\usepackage{chemfig}

\begin{document}
    \begin{day}{22/04/2021}
        \title{3}{Grupos de Hidocarbonetos de Cadeia Fechada}
        Ciclanos: São hidrocarbonetos de cadeia fechada com ligação simples antre átomos de carbono
        <image>
    \end{day}
    \begin{day}{29/04/2021}
        \title{3}{Grupos de Hidrocarbonetos de Cadeia Fechada}
        
        \textbf{Ciclenos}: são hidrocarbonetos de cadeia cíclica ou fechada com ligação dupla entre os átomos de carbono. Exemplos:
        
        \chemfig{H -[:53] C ** 5 (=[:0]C(-) - C(-[:-45]H)(-[:45]H) - C(-[:45]H)(-[:135]H) - C(-[:225]H)(-[:135]H) - C(-H)(-H))}
        
        \chemfig{[:18]() * 5 (= () - () - () - () -)}
        
        \chemfig{H_2C ** 4(-C(-[:0]H) - C(-[:0]H) - H_2C -[,,2])}
        
        \chemfig{() * 4 (- () = () - () - () -)}
        
        \chemfig{H -[:60] C ** 6 (= C(- H) - CH_2 - C(- H_2) - C(- H_2) - H_2C -)}
        
        \chemfig{[:30]() * 6 (= () - () - () - () - () -)}
        
        \title{3}{Nomeclaturas Usuais de Alguns Aromáticos}
        
        \title{3}{Atividade}
        
        Forneça a nomeclatura segundo a IUPAC para o composto a seguir:
        
        \chemfig{[:90]() * 6 (- () = () - () = ()(- CH_3) - () =)}
        Metilbenzeno
        
        \chemfig{()(-[:180]CH_3) * 6 (= () - () = ()(-[:0] CH_3) - () = () -)}
        1,4-dimetilbenzeno
        
        \title{2}{Representação Estrutural dos Hidrocarbonetos e Aplicabilidade}
        
        \title{3}{Fórmula Estrutural Plana}
        
        É a representação que mostra todos os átomos e todas as ligações presentes na molécula, pois apresenta a disposição ou arranjo dos átomos dentro de uma molécula. Exemplo:
        
        \chemfig{
              H
            - C(-[:90] H) (-[:-90] H)
            - C(-[:90] C(-[:45] H) (-[:90] H) (-[:135] H)) (-[:-90] C(-[:-45] H) (-[:-90] H) (-[:-135] H))
            - C(-[:90] H)(-[:-90] H)
            - C(-[:90] H)(-[:-90] C(-[:-45] H)(-[:-90] H) (-[:-135] H))
            - C(-[:90] H) (-[:-90] H)
            - H
        }
        
        Para esta representação, a quantidade de hidrogênios é abreviada. Exemplo:
        
        \chemfig{H_3C - C(-[:90] CH_3)(-[:-90] CH_3) - CH_2 - CH(-[:-90] CH_3) - CH_3}
        
        Trata-se da fórmula que infica o número de átomos de cada elemento em uma molécula das substâncias.
        
        \chemfig{H_3C - C(-[:90]CH_3)(-[:-90]CH_3) - CH_2 - CH(-[:-90]CH_3) - CH_3}
        \hspace{12pt}
        \chemfig{C_{8}H_{18}} --- Fórmula Molecular correspondente
        
        Fórmula estrutural plana simplificada ou condensada
        
        Assim, para determinar a fórmula molecular desse composto, basta contar a quantidade de átomos de cada elemento e colocar um índice do lado inferior direito do elemento em questão. Exemplos de Fórmula estrutural simplificada ou condensada e fórmula molecular, respectivamente:
        
        \begin{tabular}{r c}
            \chemfig{CH_3 - CH_3}             & \chemfig{C_{2}H_{6}} \\
            \chemfig{CH_3 - CH_2 - CH_3}      & \chemfig{C_{3}H_{8}} \\
            \chemfig{CH_2 = CH - CH_2 - CH_3} & \chemfig{C_{4}H_{8}}
        \end{tabular}
        
    \end{day}
    
    \begin{day}{13/05/2021}
        \title{3}{Representação dos Hidrocarbonetos}
        
        Uma das primeiras observações na Química Orgânica são as diferentes maneiras que podemos representar uma mesma molécula.
        
        \begin{center}
            \begin{tabular}{c l}
                \chemfig{() -[:30] () -[:-30] () -[:30] () -[:-30] ()} & Bastão \\[24pt]
                \chemfig{C(-[:180] H)(-[:90] H)(-[:-90] H) - C(-[:90] H)(-[:-90] H) - C(-[:90] H)(-[:-90] H) - C(-[:90] H)(-[:-90] H) - C(- H)(-[:90] H)(-[:-90] H)} & Estrutural Plana \\[50pt]
                \chemfig{H_3C - CH_2 - CH_2 - CH_2 - CH_3} & Estrutural Condensada/Simplificada \\[24pt]
                \chemfig{C_5H_{12}} & Fórmula Molecular
            \end{tabular}
        \end{center}
        
        \title{3}{Fórmula em Perspectiva}
        
        \begin{center}
            \begin{tabular}{c l}
                \chemfig{H - H} & No Plano \\
                \chemfig{H > H} & Para Trás \\
                \chemfig{H >: H} & Para Frente
            \end{tabular}
        \end{center}
        
        \title{3}{Exemplo}
        \begin{center}
            \chemfig{OH -[:30] () -[:-30] () (<[:-150] H) (<:[:-80] OH) -[:30] () (<[:120] H) (<:[:60] OH) -[:-20] () =[:20] O}
        \end{center}
        
        \title{3}{Atividade}
        
        A substância responsável pelo sabor característico da laranja apresenta a seguinte estrutura simplificada:
        
        \chemfig[atom sep=2em]{O(-[:90] C(-[:-180] C) =[:90] O) - C - C - C - C - C - C - C - C}
        
        \begin{enumerate}
            \item[a)] Qual é o número de átomos de hidrogênio presentes em uma molécula dessa substância?
            \response{20 hidrogênios}
            
            \item[b)] Qual a fórmula molecular dessa substância?
            \response{\chemfig{H_{20}C_9O_2}}
            
            \item[c)] Essa molécula possui algum heteroátomo? 
            \response{Sim.}
        \end{enumerate}
        
        A metformina é um antidiabético que pertence à classe de remédios chamados de biguanidas -- agentes utilizados como hipoglicemiantes -- ou seja, eles são capazes de reduzir/controlar a concentração de glicose na corrente sanguínea. A fórmula estrutural deste fármaco está representada a seguir.
        
        \vspace{12pt}
        \heading{2}{\chemfig[atom sep=2em, bond style={line width=1pt},bond join=true]
        {H_3C -[:30] N(-[:-90] CH_3) -[:-30] ()(-[:90] NH_2) =^[:30] N -[:-30] ()(=[:90] NH) -[:30] NH_2}}
        \vspace{12pt}
        
        \noindent Determine a fórmula molecular.
        \response{\chemfig{H_{11}C_4N_5}}
    \end{day}
\end{document}
