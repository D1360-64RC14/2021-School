\documentclass{SchoolBook}

\usepackage{amsmath}

\begin{document}
    \begin{day}{13/04/2021}
        Somente com o livro \emph{Viagem} (1939) é que Cecília Meireles ingressa no espírito poético da escola modernista. A poetista foi cuidadosa com a seleção vocabular e teve forte inclinação para a \emph{musicalidade} (traço associado ao Simbolismo), para o verso curto e para os paralelismos, a exemplo dos versos da poesia medieval portuguesa:
    
        \begin{center}
            Música \\
            \it\vspace{6pt}
            “Noite perdida  \\
            Não te lamento: \\
            embarco a vida  \\
            \vspace{6pt}
            No pensamento,   \\
            busco a alvorada \\
            do sonho isento, \\
            \vspace{6pt}
            Puro e sem nada,   \\
            -- rosa encarnada, \\
            intacta, ao vento. \\
            \vspace{6pt}
            Noite perdida,    \\
            noite encontrada, \\
            morta, vivida (...)”
        \end{center}
    \end{day}

    \begin{day}{19/04/2021 - 20/04/2021}
        \begin{enumerate}
            \item Por que no contexto histórico a Literatura Infantil é vista como uma arte abrangente?
            \response A literatura é uma arte abrangente, fenômeno de expressão que representa o Mundo, o Homem, a Vida.

            \item Da sociedade antiga até a Idade Média, como era vista a imagem da criança e sua aprendizagem?
            \response Da sociedade antiga até a Idade Média a imagem da criança era de um adulto em proporção menor; o mundo da criança era o mesmo do adulto: as crianças trabalhavam e viviam com os adultos e testemunhavam nascimentos, doenças, morte; participavam da vida pública, das festas, das guerras e de outros acontecimentos.

            \item Com a Revolução Industrial surgiram mudanças profundas na estrutura da sociedade, como a criança passa a ser vista neste período?
            \response Com a Revolução Industrial surgem mudanças profundas na estrutira da sociedade, em todos os seus segmentos, e que vão se refletir na preocupação com a infãncia. É nesse período que a criança passa, então, a ser perceida como um ser diferente do adulto, com necessidades e características próprias.

            \item No inicio do século XIX as obras conhecidas como Contos de Grimm marcaram profundamente a história da Literatura Infantil. Cite três obras das narrativas populares de Grimm que se imortalizaram no mundo todo.
            \response Branca de neve e os sete anões, João e Maria e os Músicos de Bremem.

            \item Hans Chistian Andersen foi um escritor dinamarquês autor de diversos contos infantis, dentre eles podemos destacar três obras importantes.
            \response Soldadinho de Chumbo, Patinho Feio, A Pequena Sereia.

            \item Charles Perrault publicou o Livro dia 11 de janeiro de 1697 que ficou conhecido como Contos da Mãe Gansa, nele reunia-se diversas histórias, entre elas podemos destacar três.
            \response Chapeuzinho Vermelho, a Bela Adormecida, Gato de Botas.

            \item Qual o conceito de Literatura Infantil?
            \response A Literatura Infantil é um gênero literário definido pelo público a que se destina. Certos textos são considerados pelo adultos como sendo próprios à leitera pela criança e é, a partir desse juízo, que recebem a definição de gênero e passa ma ocupar determinado lugar entre os demais livros.

            \item Cite algumas obras da autora Ruth Rocha.
            \response Marcelo, Marmelo, Martelo; O que é, o que é?; De hora em hora; Atrás da porta; O gato e a árvore; Viva a macacada.

            \item Cite algumas das obras de Chico Buarque de Holanda voltado para crianças.
            \response Estorno, Benjamin, Budapeste, Leite Derramado e O Irmão Alemão.

            \item Cite o que aconteceu com Chico Buarque durante o Período da Ditadura Militar.
            \response Durante os anos de chumbo teve várias músicas censuradas e foi ameaçado, tendo se exilado na Itália em 1969.

            \item Sobre o que fala o livro Menina Bonita do Laço de Fita de Ana Maria Machado?
            \response Valorização, respeito mútuo, respeito sobre as diferentes etnias, etc.

            \item Quais as principais características da obras de Lygia Bojunga Nunes?
            \response Principais características das obras de Lígia: ela escreve livros para crianças retratando os problemas da vida social de forma "leve" (de forma que a criança possa entender) e aprende a lidar com os problemas sociais, como a fome, abandono, diversidade de gêneros, etc.

            \item Quais as características da obra de Cecília Meireles?
            \response Filia-se às tradções da lírica luso-brasileira. Apesar disso, suas publicações iniciais evidenciam certa inclinação pelo Simbolismo, reúnem religiosidade, desespero e individualizamo. Há misticismo no campo da solidão, mas existe a consciencia de seus dons e deu destino. Uso de elementos como o vento, a água, o mar, o ar, o tempo, o espaço, a solidão e a música, confirmam a inclinação neossimbolista.
        \end{enumerate}
    \end{day}
    
    \begin{day}{28/06/2021}
        
    \end{day}
\end{document}

https://youtu.be/jJWoNYUtjqI?t=1766
    
YKURmO9bnR0
NBYh3kYx2Io
