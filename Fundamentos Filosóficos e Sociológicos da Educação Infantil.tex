\documentclass{SchoolBook}

\begin{document}
    \begin{day}{04/05/2021}
        Conte afirmou ter descoberto uma grande lei fundamental, segunda a qual o espírito humano em sua evolução passou por três estados:

        \vspace{6pt}
        \begin{itemize}[nosep]
            \item Estado teológico;
            \item Estado metafísico;
            \item Estado positivista.
        \end{itemize}

        \title{3}{Estado teológico}
        O espírito humano enconta-se nos agentes sobrenaturais a explicação dos fenômenos. No estado teológico a explicação dos fenômenos supõe uma casualidade sobrenatural. Os fenômenos da natureza, a origem dos seres, e os costumes são explicadospela ação dos deuses.

        \title{3}{Estado metafísico}
        Os fenômenos são explicados não mais por agentes sobrenaturais (fetichismo, politeísmo e monismo), mas por forças abstratas.

        \title{3}{Estado positivista}
        Este é o último e definitivo, onde encotnra a ciência e é deixado de lado a investigação das causas primeiras ou finais, se atem a observação dos fatos procurando raciocinar sobre eles e descobrir as relações constantes entre os fenômenos observáveis, isto é, suas leis.
        
        Para Comte, o tempo "positivo" designa o real em oposição a quimérico, a certeza em oposição à indecisão, o preciso em oposição ao vago.
        
        Portanto, o estado Positivo corresponde à maturidade do espírito humano, objetivo de toda educação daí em frente.
    \end{day}
    
    \begin{day}{05/05/2021}
        \begin{enumerate}
            \item[1.] Qual o principal objetivo de Augusto Conte para a educação?
            \response{Educação exclusivamente baseada na ciência, e um objetivo principal de promover o altruísmo e repreender o egoísmo.}
        \end{enumerate}
    \end{day}
    
    \begin{day}{19/05/2021}
        \title{3}{A função da Escola}
        
        Segundo Durkheim, a educação éa ação ecercida pelas gerações adultas sobre aquelas que ainda não estão maduras para a vida social.
        
        Através da educação o "ser individual" transforma-se em "ser social". Essa socialização se opera desde o nascimento, na família, porém é na escola que é sistematizada.
        
        \title{3}{Qual a Pedagogia?}
        
        Os elementos da moralidade que definem as metas que a sociologia da educação fixa para a escola:
        
        \begin{itemize}[nosep]
            \item Espírito da disciplina
            \item Vinculação aos grupos
            \item Autonomia da vontade
        \end{itemize}
        
        \title{3}{A relação Mestre-aluno}
        \title{4}{Fatores necessários ao mestre}
        
        \begin{itemize}[nosep]
            \item Autoridade
            \item Suscitar o respeito por parte dos alunos
            \item "Cultura psicológica"
        \end{itemize}
        
        \title{4}{O Meio-Escolar}
        
        \begin{itemize}[nosep]
            \item Mais extensa que a família
            \item Senso de grupo
            \item Definição grupo-classe
            \item O papel do mestre
        \end{itemize}
        
        \title{4}{O poder do mestre}
        
        \begin{itemize}[nosep]
            \item Espírito de disciplina
            \item Colonizador - colonizado
            \item Poder do mestre e o seu alcance
        \end{itemize}
        
        \title{4}{O Poder dos Saberes}
        
        \emph{"A escola é não somente um lugar de `educação', em particular, de educação `moral', mas também e, ao mesmo tempo, um lugar de `instrução', de aquisição de saberes."} (FILLOUX, 2010)
    \end{day}
    
    \begin{day}{26/05/2021}
        
    \end{day}
\end{document}
