\documentclass{SchoolBook}

\begin{document}
    \begin{day}{04/05/2021}
        Conte afirmou ter descoberto uma grande lei fundamental, segunda a qual o espírito humano em sua evolução passou por três estados:

        \vspace{6pt}
        \begin{itemize}[nosep]
            \item Estado teológico;
            \item Estado metafísico;
            \item Estado positivista.
        \end{itemize}

        \title{3}{Estado teológico}
        O espírito humano enconta-se nos agentes sobrenaturais a explicação dos fenômenos. No estado teológico a explicação dos fenômenos supõe uma casualidade sobrenatural. Os fenômenos da natureza, a origem dos seres, e os costumes são explicadospela ação dos deuses.

        \title{3}{Estado metafísico}
        Os fenômenos são explicados não mais por agentes sobrenaturais (fetichismo, politeísmo e monismo), mas por forças abstratas.

        \title{3}{Estado positivista}
        Este é o último e definitivo, onde encotnra a ciência e é deixado de lado a investigação das causas primeiras ou finais, se atem a observação dos fatos procurando raciocinar sobre eles e descobrir as relações constantes entre os fenômenos observáveis, isto é, suas leis.
        
        Para Comte, o tempo "positivo" designa o real em oposição a quimérico, a certeza em oposição à indecisão, o preciso em oposição ao vago.
        
        Portanto, o estado Positivo corresponde à maturidade do espírito humano, objetivo de toda educação daí em frente.
    \end{day}
    
    \begin{day}{05/05/2021}
        \begin{enumerate}
            \item[1.] Qual o principal objetivo de Augusto Conte para a educação?
            \response{Educação exclusivamente baseada na ciência, e um objetivo principal de promover o altruísmo e repreender o egoísmo.}
        \end{enumerate}
    \end{day}
    
    \begin{day}{11/05/2021}
        \title{3}{Os Clássicos da Sociologia}
        
        
    \end{day}
\end{document}
