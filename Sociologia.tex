\documentclass{SchoolBook}

\begin{document}
    \begin{day}{23/04/2021}
        \begin{enumerate}
            \item[1.] Explique, com suas palavras, os principais postulados do liberalismo, a partir das ideias de Locke, Smith e Montesquieu. Escreva um parágrafo em seu caderno.
            \response Entre os postulados do liberalismo, destacam-se a ideia de contrato social e o direito à vida, à liberdade e à propriedade privada (Locke), as ideias do liberalismo econômico de Adam Smith, no qual indivíduos interessados impulsionam o crescimento econômico, o lucro e as inovações tecnológicas.
            Por sua vez, a ideia de divisão de poderes, defendida por Montesquieu, também é outra base do liberalismo.

            \item[2.] Cite e explique ao menos 3 características do nazi-fascismo. Escreva um parágrafo em seu caderno.
            \response Você deve enfatizar que os pontos em comum do fascismo e do nazismo são numerosos. Entre as características, podemos citar o totalitarismo, ou o controle total sobre amplos aspectos da sociedade (política, cultura, religião, pensamento, etc) e o desrespeito aos direitos humanos.
        \end{enumerate}
    \end{day}
    
    \begin{day}{07/05/2021}
        \begin{enumerate}
            \item[1.] Qual foi o contexto de surgimento do Estado de Bem-Estar Social? Escreva um parágravo em seu caderno.
            \response{Sua resposta deve considerar contexto pós-crise capitalista de 1929. Ela exigiu a INTERVENÇÃO do Estado na Economia, para combater o desemprego e a inflação. Esse contexto também exigiu uma resposta ao crescimento de ovimentos operários e regimes anti-liberais.}
            
            \item[2.] O Estado de Bem-Estar Social, assim como foi definido, se desenvolveu após a Segunda Guerra Mundial, tendo como importante referência o Keynesianismo, com o intuito de reverter os efeitos nefastos da crise econômica de 1929.
            
            O Estado de Bem-Estar social possui as mesmas premissas do Estado Liberal?
            \response{Sua resposta deve considerar que o Estado de Bem-Estar Social e o Estado Liberal são capitalistas. No entanto, o Estado de Bem-Estar Social possui encaminhamentos diferentes do Estado Liberal, pois possui como foco a intervenção na economia e a promoção do Bem-Estar da população no campo social (Saúde, Educação, Emprego, etc).}
        \end{enumerate}
    \end{day}
\end{document}
