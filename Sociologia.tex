\documentclass{SchoolBook}

\begin{document}
    \begin{day}{23/04/2021}
        \begin{enumerate}
            \item[1.] Explique, com suas palavras, os principais postulados do liberalismo, a partir das ideias de Locke, Smith e Montesquieu. Escreva um parágrafo em seu caderno.
            \response Entre os postulados do liberalismo, destacam-se a ideia de contrato social e o direito à vida, à liberdade e à propriedade privada (Locke), as ideias do liberalismo econômico de Adam Smith, no qual indivíduos interessados impulsionam o crescimento econômico, o lucro e as inovações tecnológicas.
            Por sua vez, a ideia de divisão de poderes, defendida por Montesquieu, também é outra base do liberalismo.

            \item[2.] Cite e explique ao menos 3 características do nazi-fascismo. Escreva um parágrafo em seu caderno.
            \response Você deve enfatizar que os pontos em comum do fascismo e do nazismo são numerosos. Entre as características, podemos citar o totalitarismo, ou o controle total sobre amplos aspectos da sociedade (política, cultura, religião, pensamento, etc) e o desrespeito aos direitos humanos.
        \end{enumerate}
    \end{day}
    
    \begin{day}{30/04/2021}
        
    \end{day}
\end{document}
