\documentclass{SchoolBook}

\begin{document}
    \begin{day}{04/05/2021}
        \title{2}{
            Docência na educação infantil: das rotinas às vidas cotidiana. \\
            Professor Altino José Martins Filho. \\
            Universidade do Estado de Santa Catarina.
        }

        \title{3}{Problematização Assunto do Vídeo}

        \begin{itemize}[nosep]
            \item Trabalhar a visibilidade da idade da criança e suas formas de interação observar a concepção de infância que norteia o trabalho;

            \item Entender que a docência na educação infantil não se faz sem as crianças;

            \item Abrir espaço para o desenvolvimento da infância e não só da escolarização;
            
            \item Perceber que a docência na educação infantil se faz de forma coletiva.
        \end{itemize}
    \end{day}
    
    \begin{day}{Reflexão sobre a prática pedagógica na Educação Infantil --- 11/05/2021}
        \title{Video da Professora Katia Stoco Smole}
        
        \begin{itemize}[nosep]
            \item Resolução de problemas como possibilidade para o desenvolvimento do raciocínio lógico matemático;
            \item Necessidade de criar problemas para a criança resolver;
            \item Possibilidades de desenvolvimento -- Trabalhar a resiliência, a testagem de hipóteses, o começar de novo, a desnesessidade de chegar a um resultado previsto, entre outros.
        \end{itemize}
        
        Oração
        Assunto muito avançado
        Leitura de texto sem ser direta para as crianças
        Passos relativamente tecnicos
        Tesoura com ponta
        (Objetos e materiais cortantes)
        Assunto sendo passado muito rápido
        Objetos que quebram com facilidade (trocar prato por tampa)
        Atividade complicada
        Jogo da velha sem explicação das regras 
        
    \end{day}
\end{document}
