\documentclass{SchoolBook}

\begin{document}
    \begin{day}{04/05/2021}
        \title{2}{
            Docência na educação infantil: das rotinas às vidas cotidiana. \\
            Professor Altino José Martins Filho. \\
            Universidade do Estado de Santa Catarina.
        }

        \title{3}{Problematização Assunto do Vídeo}

        \begin{itemize}[nosep]
            \item Trabalhar a visibilidade da idade da criança e suas formas de interação
            \item Observar a concepção de infância que norteia o trabalho;
            \item Entender que a docência na educação infantil não se faz sem as crianças;
            \item Abrir espaço para o desenvolvimento da infância e não só da escolarização;
            \item Perceber que a docência na educação infantil se faz de forma coletiva.
        \end{itemize}
    \end{day}
    
    \begin{day}{Reflexão sobre a prática pedagógica na Educação Infantil --- 11/05/2021}
        \title{Video da Professora Katia Stoco Smole}
        
        \begin{itemize}[nosep]
            \item Resolução de problemas como possibilidade para o desenvolvimento do raciocínio lógico matemático;
            \item Necessidade de criar problemas para a criança resolver;
            \item Possibilidades de desenvolvimento -- Trabalhar a resiliência, a testagem de hipóteses, o começar de novo, a desnesessidade de chegar a um resultado previsto, entre outros.
        \end{itemize}
    \end{day}
    
    \begin{day}{18/05/2021}
        \begin{enumerate}
            \item[1.] O que você compreende por Prática Pedagógica?
            \response{A Prática Pedagógica é o ato de colocar em prática tudo o que a pedagogia aborda.}
            
            \item[2.] O que você compreende por Práxis Pedagógia?
            \response{}
            
            \item[3.] Explique a afirmação "toda a Práxis é atividade, mas nem toda atividade é Práxis".
        \end{enumerate}
    \end{day}
    
    \begin{day}{25/05/2021}
        \title{3}{Video: A educação infantil na BNCC}
        
        \begin{itemize}[nosep]
            \item Eixos estruturantes da prática pedagógica
            \item Interação
            \item Brincadeiras
        \end{itemize}
        \vspace{6pt}
        
        Esses eixos dão a estrutura do trabalho de ensino e aprendizagem a ser desenvolvido pelo professor.
        
        \title{3}{Direitos da Criança na Educação Infantil}
        
        \begin{itemize}[nosep]
            \item O Brincar
            \item O Conviver
            \item O Expressar
            \item O Conhecer (Identidade cultural, individual, e social)
            \item Participar
            \item Explorar
        \end{itemize}
        
        \title{3}{Campos de Experiências}
        
        \begin{enumerate}[nosep]
            \item[1º -] O eu, o outro, e o nós
            \item[2º -] Corpo, gesto, e movimentos
            \item[3º -] Traços, sons, cores, e formas
            \item[4º -] Escutar, falar -- Oralidade
            \item[5º -] Espaços, tempos, quantidades, e transformações
        \end{enumerate}
        
        No primeiro trimestre do ano de 2021 foram assistidas aulas ministradas pela professora Alzenira na disciplina de Prática de Formação e interação com os cursistas. Nestas aulas foram realizadas apresentações de vídeos, aulas expositivas, demonstrações teóricas sobre os assuntos referentes a Educação Infantil. As aulas ministradas pela professora ocorreram por meio de apressentações de slides, confecção de materiais pedagógicos por parte dos cursistas, produção de textos, planejamentos, observação de aulas apresentadas por outros professores.
        
        
    \end{day}
\end{document}

