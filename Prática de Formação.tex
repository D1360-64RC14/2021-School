\documentclass{SchoolBook}

\begin{document}
    \begin{day}{04/05/2021}
        \title{2}{
            Docência na educação infantil: das rotinas às vidas cotidiana. \\
            Professor Altino José Martins Filho. \\
            Universidade do Estado de Santa Catarina.
        }

        \title{3}{Problematização Assunto do Vídeo}

        \begin{itemize}[nosep]
            \item Trabalhar a visibilidade da idade da criança e suas formas de interação
            \item Observar a concepção de infância que norteia o trabalho;
            \item Entender que a docência na educação infantil não se faz sem as crianças;
            \item Abrir espaço para o desenvolvimento da infância e não só da escolarização;
            \item Perceber que a docência na educação infantil se faz de forma coletiva.
        \end{itemize}
    \end{day}
    
    \begin{day}{Reflexão sobre a prática pedagógica na Educação Infantil --- 11/05/2021}
        \title{Video da Professora Katia Stoco Smole}
        
        \begin{itemize}[nosep]
            \item Resolução de problemas como possibilidade para o desenvolvimento do raciocínio lógico matemático;
            \item Necessidade de criar problemas para a criança resolver;
            \item Possibilidades de desenvolvimento -- Trabalhar a resiliência, a testagem de hipóteses, o começar de novo, a desnesessidade de chegar a um resultado previsto, entre outros.
        \end{itemize}
    \end{day}
    
    \begin{day}{18/05/2021}
        \begin{enumerate}
            \item[1.] O que você compreende por Prática Pedagógica?
            \response{A Prática Pedagógica é o ato de colocar em prática tudo o que a pedagogia aborda.}
            
            \item[2.] O que você compreende por Práxis Pedagógia?
            \response{}
            
            \item[3.] Explique a afirmação "toda a Práxis é atividade, mas nem toda atividade é Práxis".
        \end{enumerate}
    \end{day}
    
    \begin{day}{25/05/2021}
        \title{3}{Video: A educação infantil na BNCC}
        
        \begin{itemize}[nosep]
            \item Eixos estruturantes da prática pedagógica
            \item Interação
            \item Brincadeiras
        \end{itemize}
        \vspace{6pt}
        
        Esses eixos dão a estrutura do trabalho de ensino e aprendizagem a ser desenvolvido pelo professor.
        
        \title{3}{Direitos da Criança na Educação Infantil}
        
        \begin{itemize}[nosep]
            \item O Brincar
            \item O Conviver
            \item O Expressar
            \item O Conhecer (Identidade cultural, individual, e social)
            \item Participar
            \item Explorar
        \end{itemize}
        
        \title{3}{Campos de Experiências}
        
        \begin{enumerate}[nosep]
            \item[1º -] O eu, o outro, e o nós
            \item[2º -] Corpo, gesto, e movimentos
            \item[3º -] Traços, sons, cores, e formas
            \item[4º -] Escutar, falar -- Oralidade
            \item[5º -] Espaços, tempos, quantidades, e transformações
        \end{enumerate}
        
        No primeiro trimestre do ano de 2021 foram assistidas aulas ministradas pela professora Alzenira na disciplina de Prática de Formação e interação com os cursistas. Nestas aulas foram realizadas apresentações de vídeos, aulas expositivas, demonstrações teóricas sobre os assuntos referentes a Educação Infantil. As aulas ministradas pela professora ocorreram por meio de apressentações de slides, confecção de materiais pedagógicos por parte dos cursistas, produção de textos, planejamentos, observação de aulas apresentadas por outros professores.
        
        No dia dezoito de Maio, último dia do primeiro trimestre, a professora Alzemira pediu para que seja realizado o relatório de atividades executadas, falando sobre os assuntos debatidos no trimestre. Sendo eles "função social e cultural da escola", "o brincar e o cuidar na educação infantil", "estudo e planejamento na educação infantil" e "práxis pedagógica".
        
        A função social da escola é o desenvolvimento das potencialidades físicas, cognitivas e afetivas do indivíduo, capacitando-o a se tornar um cidadão participativo na sociedade em que vive.
        
        A função cultural da escola é abrir a mente do educando para realidades ou pontos de vista diferentes do dele. Assim podendo fazê-lo se sentir participante do processo educacional ou introduzindo-o a esse. O educando aprende a importância de cada cultura e da existência de variados costumes.
        
        Quando criança, o brincar é um ato não apenas divertido, mas também muito importante. A brincadeira auxilia na formação, socialização, desenvolvimento de habilidades psicomotoras, sociais, físicas, afetivas, cognitivas e emocionais. Ao brincar a criança expõe seus sentimentos, aprende, constroem, exploram, pensam, sentem, reinventam e se movem. Fantasiando a criança revive angústias, conflitos, alegrias, desiste e refaz, deixando de lado a sujeição as ordens e exigências dos adúltos, inserindo-se na sociedade onde assimilam valores, crenças, leis, regras, hábitos, cosumes, princípios e linguagens. As crianças são capazes de lidar com complexas dificuldades psicológicas ao brincar.
        
        Educar destaca-se no desenvolvimento das faculdades físicas, morais e intelectuais dos indivíduos indo em direção à preocupação com o intelectual, cognitivo, moral e social. É proporcionar junto ao cuidar maneiras que auxiliarão o aprendizado.
        
        Cuidar não é apenas estar atento a momentos e situações que podem causar risco à criança.
        
        Planejamento é muito importante para a organização e melhor entendimento da aula. É ao planejar que o professor: define o tempo da aula, e da execução de cada tarefa; planeja como será ou como organizará o espaço, os materiais, o agupamento das crianças, etc. E para tudo isso é necessário estudo para compreenção de como executar o planejamento da melhor forma possível. Pesquisando e trocando ideia com pessoas de maior conhecimento ou até do local.
        
        
        
        
        [x] FUNÇÃO SOCIAL E CULTURAL DA ESCOLA
        [x] O BRINCAR E O CUIDAR NA EDUCAÇÃO INFANTIL
        [x] ESTUDO E PLANEJAMENTO NA EDUCAÇÃO INFANTIL
        [ ] REFLEXÃO SOBRE A PRÁXIS PEDAGÓGICA
    \end{day}
    
    \begin{day}{08/06/2021}
        Video: Maria Campos Malta -- PUC São Paulo: O perfil do professor de Educação Infantil.
        
        \begin{itemize}[nosep]
            \item A observação da mudança do ambiente pela criança muito pequena: doméstico e o social.
            \item A reflexão sobre as considerações do funcionamento de uma instituição de educação infantil. O que é funcionar bem?
            \item Entender o processo de ensino na educação infantil por meio da compreensão sobre o desenvolvimento da criança e suas aprendizagens -- Interações e brincadeiras.
            \item Compreender as dificuldades que estão atreladas ao ensinar e como ensinar.
            \item A valorização do profissional -- É importante observar que quanto menor a criança, menor o prestígio do profissional.
            \item Observar a especificidade na organização do trabalho pedagógico e no encaminhamento dos conteúdos da educação infantil.
            \item A professora precisa saber o que precisa ser feito com a criança. Ela precisa ser ajudado para isso.
        \end{itemize}
        
        \begin{enumerate}[nosep]
            \item[1.] Proposta de Plano de Aula.
            \item[2.] Desenvolver um plano de aula que comtemple a história.
        \end{enumerate}
        
        \title{4}{Estrutura do Planejamento}
        
        \noindent Turma: \\
        Numero de Alunos: \\
        Pofessor(a):
        
        \begin{enumerate}
            \item[a)] Objetivo Geral;
            \item[b)] Objetivo Específico:
            \begin{itemize}[nosep]
                \item[*] Expressar a opinião sobre os acontecimento ocorridos na história;
                \item[*] Explora a oraldiade das crianças.
            \end{itemize}
            
            \item[c)] Encaminhamentos metodológicos -- (Descrição dos procedimentos que você irá realizar desde o momento que inicirá o conto da história, as atividades que irá trabalhar para atender os objetivos específicos);
            
            \item[d)] Recursos didáticos;
            \item[e)] Referências pesquisadas;
            \item[f)] Avaliação.
        \end{enumerate}
    \end{day}
\end{document}

