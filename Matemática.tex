\documentclass{SchoolBook}

\begin{document}
    \begin{day}{12/04/2021}
        \title{3}{Raio}

        Distância do centro (origem) até a circunferência.

        \title{3}{Diâmetro}

        Distância de um ponto até outro da circunferência passando pelo centro. (dobro do raio).

        \title{3}{Corda}

        Distância entre dois pontos da circunferência que não passa pelo centro.

        \title{3}{Altura do sólido}

        Distância  entre as bases formando um ângulo de 90°.

    \end{day}
      
    \begin{day}{04/05/2021}
        $$ circunferenciaBase = \pi * r $$
        $$ areaBase = \pi * r^2 $$
    
        \begin{enumerate}
            \item[251.] \textbf{Altenrativa (a)}
            
            \begin{math}
                circunferenciaBase = \pi * 2 = 2\pi \\
                areaLateral = 2\pi * 3 = 6\pi
            \end{math}
            
            \item[252.] \textbf{Alternativa (b)}
            
            \begin{math}
                areaBase = \pi * (\frac{2m}{2})^2 = 3,14 m^2 \\
                areaSolido = 3,14 m^2 * 0,7 m = 2,198 m^3
            \end{math}
            
            \item[253.] \textbf{Alternativa (d)}
            
            \begin{math}
                areaBase = \pi * (\frac{2mm}{2})^2 = 3,14 mm^2 \\
                areaSolido = 31,4 cm^2 * 12cm = 376,8 cm^3 = \frac{376,8}{2} = 37,68 mm^3
            \end{math}
            
            \item[254.] \textbf{Alternativa (c)}
            
            \begin{math}
                areaBase = \pi * (3m)^2 = 9,42 m^2 \\
                areaSolido = 9,42 m^2 * 10m = 94,2 m^3 \\
                tempo = \frac{94,2 m^3}{5 h} = 18,84 m^3 / h
            \end{math}
            
            \begin{math}
                areaBase = \pi * (3m)^2 = 3\pi m^2 \\
                areaSolido = 3\pi m^2 * 10m = 30\pi m^3 \\
                tempo = \frac{30\pi m^3}{5 h} = 6 \pi m^3 / h
            \end{math}
            
        \end{enumerate}
    \end{day}
    
    \begin{day}{26/07/2021}
        \title 3 {Elementos de um Poliedro}
        
        \begin{itemize}
            \item Face: é cada uma das superfícies poligonais que compõem a superfície do poliedro.
            \item Aresta: lado comum a dias faces.
            \item Vértice: ponto comum a três ou mais arestas.
        \end{itemize}
    \end{day}
    
    \begin{day}
        \title 3 {Relação e Euler}
        
        $$ V + F = A + 2 $$
        
        Sendo:
        
        \begin{tabular}{c l}
            V & = Vértices \\
            A & = Arestas  \\
            F & = Faces
        \end{tabular}
    \end{day}
    
    \begin{day}{03/08/2021}
        Sólidos de platão:
        
        Tetraedro (fogo): 4 faces triangulares iguais
        Hexaedro (terra): 6 faces quadirangulares iguais
        Octaedro (ar): 8 faces triangulares iguais
        Icosaedro (água): 20 faces triangulares iguais
        Dodecaedro (universo): faces pentagonais
    \end{day}
\end{document}
