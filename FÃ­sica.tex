\documentclass{SchoolBook}

\usepackage{lipsum}

\begin{document}
    \begin{day}{22/04/2021}
        \begin{enumerate}
            \item[1.] Sobre um corpo de massa $ 8 kg $, inicialmente em repouso, age uma força constante $ \vec{F} = 80 N $, na direção do deslocamento.
            Determine o trabalho realizado pela foça nos primeiros 20 segundos de movimento.

            \item[2.] Um ponto material de massa $ 6 kg $ tem velocidade de $ 8 m/s $ quando sobre ele passa a agir uma força de intensidade $ 30 N $ na direção do movimento, durante $ 4 s $. Determine:
            \begin{enumerate}
                \item[a)] o deslocamento durante esses $ 4 s$;
                \item[b)] o trabalho realizado nesse deslocamento.
            \end{enumerate}

            \item[3.] Um móvel de massa $ 40 kg $ tem velocidade constante de $ 90 km/h $.
            Num determinado instante entra numa região rugosa onde o coeficiente de atrito é igual a $ 0,2 $. Determine:
            \begin{enumerate}
                \item[a)] o espaço percorrido pelo móvel na região rugosa, até parar;
                \item[b)] o trabalho realizado pela força de atrito.
            \end{enumerate}
        \end{enumerate}
    \end{day}
\end{document}