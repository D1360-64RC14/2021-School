\documentclass{SchoolBook}

\usepackage{multicol}

\begin{document}
    \begin{day}{22/04/2021}
        \begin{enumerate}
            \item[1.] Complete os espaços nas sentenças abaixo com a conjugação correta no \textbf{Simple Past Tense -- Passado Simples}.
            Se necessário consulte a página \textbf{Simples Past Tense -- Verbos Irregulares}.
            Caso os verbos não sejam irregulares adicione "ed" conforme está na página de \textbf{Simples Past Tense -- Frases Afirmativas}.

            Ex: Jane \emph{(love)} \underline{\bf loved} the concert last night! (Jane amou o concerto, ontem à noite!)
            \begin{enumerate}
                \item[a)] Yesterday, Ronald \emph{(go)} \underline{\bf went} to school by car. (Ontem, Ronald foi para a escola de carro.)
                \item[b)] Sam \emph{(have)} \underline{\bf had} breakfast early today. (Sam tomou café da manhã cedo hoje.)
                \item[c)] Juliet \emph{(study)} \underline{\bf studied} in the University of California. (Juliet estudou na Universidade da Califórnia)
                \item[d)] I \emph{(eat)} \underline{\bf ate} lasagna last Sunday. (Eu comi lasanha no domingo passado.)
                \item[e)] My friend \emph{(arrive)} \underline{\bf arrived} a 10:00 p.m. last night. (Meu amigo chegou às 10h00 ontem à noite.) 
            \end{enumerate}

            \item[2.] Fill in the \textbf{Regular} verbs in parentheses into the sentences. Use the Simple Past Tense.
            \begin{enumerate}
                \item[a)] Lico \underline{\bf paid} soccer yesterday. (to play)
                \item[b)] Sarah \underline{\bf watched} a cartoon on TV. (to watch)
                \item[c)] They \underline{\bf carried} their books. (to carry)
                \item[d)] He \underline{\bf helped} his father last Saturday. (to help)
                \item[e)] John and Mary \underline{\bf cleaned} the house. (to clean)
            \end{enumerate}

            \item[3.] Fill in the \textbf{Irregular} verbs in parentheses into the sentences. Use the Simple Past Tense.
            \begin{enumerate}
                \item[a)] She \underline{\bf drove} a beautiful car last week. (to drive)
                \item[b)] Frank \underline{\bf ate} the whole pizza. (to eat)
                \item[c)] They \underline{\bf had} a very good teacher last year. (to have)
                \item[d)] The candidates \underline{\bf wrote} their opinion on the paper. (to write)
                \item[e)] He \underline{\bf sang} his favorite song. (to sing)
            \end{enumerate}

            \item[4.] Complete as sentenças com o Passado Simples dos verbos dados.
            \begin{enumerate}
                \item[a)] The cat \underline{\bf jumped} the fence. (to jump)
                \item[b)] The children \underline{\bf missed} the school bus. (to miss)
                \item[c)] You \underline{\bf wrote} a beautiful poem. (to write)
                \item[d)] My friends \underline{\bf tried} to understand Arabic language. (to try)
                \item[e)] They \underline{\bf studied} Math in their rooms. (to study)
            \end{enumerate}

            \pagebreak

            \item[5.] Traduza (para o português) as orações do exercício 4.
            \begin{enumerate}
                \item[a)] \_\_\_\_\_\_\_\_\_\_\_\_\_\_\_\_\_\_\_\_\_\_\_\_\_\_\_\_\_
                \item[b)] \_\_\_\_\_\_\_\_\_\_\_\_\_\_\_\_\_\_\_\_\_\_\_\_\_\_\_\_\_
                \item[c)] \_\_\_\_\_\_\_\_\_\_\_\_\_\_\_\_\_\_\_\_\_\_\_\_\_\_\_\_\_
                \item[d)] \_\_\_\_\_\_\_\_\_\_\_\_\_\_\_\_\_\_\_\_\_\_\_\_\_\_\_\_\_
                \item[e)] \_\_\_\_\_\_\_\_\_\_\_\_\_\_\_\_\_\_\_\_\_\_\_\_\_\_\_\_\_
            \end{enumerate}

            \item[6.] Agora, neque todas as orações do exercício 4.
            \begin{enumerate}
                \item[a)] \_\_\_\_\_\_\_\_\_\_\_\_\_\_\_\_\_\_\_\_\_\_\_\_\_\_\_\_\_
                \item[b)] \_\_\_\_\_\_\_\_\_\_\_\_\_\_\_\_\_\_\_\_\_\_\_\_\_\_\_\_\_
                \item[c)] \_\_\_\_\_\_\_\_\_\_\_\_\_\_\_\_\_\_\_\_\_\_\_\_\_\_\_\_\_
                \item[d)] \_\_\_\_\_\_\_\_\_\_\_\_\_\_\_\_\_\_\_\_\_\_\_\_\_\_\_\_\_
                \item[e)] \_\_\_\_\_\_\_\_\_\_\_\_\_\_\_\_\_\_\_\_\_\_\_\_\_\_\_\_\_
            \end{enumerate}
        \end{enumerate}
    \end{day}
    
    \begin{day}{28/04/2021}
        \title{3}{A estrutura do \emph{Present Perfect}}
        
        \vspace{3pt}
        
        \textbf{Para as afirmativas}: \colorx{green}{SUJEITO} + \colorx{red}{PRESENTE SIMPLES DO VERBO \emph{TO HAVE}} + PARTICÍPIO PASSADO DO VERBO PRINCIPAL. \\
        \colorx{green}{I} \colorx{red}{have} tried sushi before.
        
        \vspace{3pt}
        
        \textbf{Para as negativas}: \colorx{green}{SUJEITO} + \colorx{red}{PRESENTE SIMPLES DO VERBO \emph{TO HAVE}} + \colorx{red}{\emph{NOT}} + PARTICÍPIO PASSADO DO VERBO PRINCIPAL. \\
        \colorx{green}{I} \colorx{red}{have not} tried sushi before.
        
        \vspace{3pt}
        
        \textbf{Para as interrogativas}: \colorx{red}{PRESENTE SIMPLES DO VERBO \emph{TO HAVE}} + \colorx{green}{SUJEITO} + PARTICÍPIO PASSADO DO VERBO PRINCIPAL + ? \\
        \colorx{red}{Have} \colorx{green}{you} tried sushi before?
        
        \vspace{12pt}
        \noindent Utiliza-se o \emph{present perfect} para descreve ações que iniciam no passado e se estendem até o presente. \\
        Exemplos:
        
        \emph{I have \colorx{red}{studied} at school since \colorx{blue}{7 o'clock}}. (Eu tenho estudado na escola desde às sete horas)
        
        \emph{They have \colorx{red}{learned} about music since last year}. (Eles têm aprendido sobre música desde o ano passado)
        
        \vspace{12pt}
        \noindent Utiliza-se o \emph{present perfect} para descrever ações que vêm ocorrendo recentemente. \\
        Exemplos:
        
        \emph{They have \colorx{red}{been} sad recently}. (Eles têmestado tristes recentemente)
        
        \emph{We \colorx{red}{have} run every day in the park}. (Nós temos corrido todos os dias no parque)
        
        \vspace{12pt}
        \noindent Utiliza-se o \emph{present perfect} para descrever ações que acabaram de ocorrer. \\
        Exemplos:
        
        \emph{We \colorx{red}{have} just finished our work}. (Nós acabamos de terminar nosso trabalho)
        
        \emph{She \colorx{red}{has} just read the magazine}. (Ela acabou de ler a revista)
        
        \vspace{12pt}
        \noindent Utiliza-se o \emph{present perfect} para descrever ações que ocorreram em um momento indefinido do passado. \\
        Exemplos:
        
        \emph{You \colorx{red}{have} played video game for a long time}. (Você têm jogado vídeo game por um longo tempo)
        
        \emph{I \colorx{red}{have} cooked for you}. (Eu tenho cozinhado para vocês)
        
        \vspace{12pt}
        \noindent Formação das frases na negativa e na interrogativa:
        
        \noindent O negativo do \emph{present perfect} é formado pelo acréscimo do \textbf{"not"} na frase (forma contraída: \textbf{haven't/hasn't})
        
        \begin{enumerate}
            \item[1.] Supply the present perfect tense od the given: \\
            Example:
            
            I \underline{have} already \underline{written} the book. (write)
            
            \begin{enumerate}[nosep]
                \item[a)] He \underline{\bf has worked} here for three months. (work)
                \item[b)] They \underline{\bf have gone} to Australia many times. (go)
                \item[c)] \underline{\bf Have} you ever \underline{\bf went} to Paris? (be)
                \item[d)] My family \underline{\bf have visited} me recently. (visit)
                \item[e)] My parents \underline{\bf haven't arrived} yet. (not arrive)
            \end{enumerate}
        \end{enumerate}
    \end{day}
    
    \begin{day}{29/04/2021}
        \begin{enumerate}
            \item[3.] Fill in:
            \begin{enumerate}[nosep]
                \item[a)] What {\bf\underline{has it}} done? {\bf\underline{It drunk}} the milk. (drink)
                \item[b)] What's he done? {\bf\underline{He has cleaned}} the windows. (clean)
                \item[c)] What have they done? {\bf\underline{They have broken}} the window. (break)
            \end{enumerate}
        \end{enumerate}
        
        \begin{multicols}{2}
            \title{3}{A long way to travel}
        
            Alex and his mother were talking on the beach one day after a storm.
            
            "Look", Alex said. He pointed to a piece of round, vlue glass almost covered with sand. They dug around it and found a round, glass ball. It was bigger than a softball, but smaller thana soccer ball.
            
            "What is it?" Alex asked.
            
            \vfill\columnbreak
            
            \title{3}{Uma viagem distante}
            
            %Alex e sua mãe estavam conversando na praia após o temporal.
        
        \end{multicols}
        
    \end{day}
\end{document}
