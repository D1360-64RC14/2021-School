\documentclass{SchoolBook}

\usepackage{multicol}
\setlength\columnsep{28pt}

\begin{document}
    \begin{day}{22/04/2021}
        \begin{enumerate}
            \item[1.] Complete os espaços nas sentenças abaixo com a conjugação correta no \textbf{Simple Past Tense -- Passado Simples}.
            Se necessário consulte a página \textbf{Simples Past Tense -- Verbos Irregulares}.
            Caso os verbos não sejam irregulares adicione "ed" conforme está na página de \textbf{Simples Past Tense -- Frases Afirmativas}.

            Ex: Jane \emph{(love)} \underline{\bf loved} the concert last night! (Jane amou o concerto, ontem à noite!)
            \begin{enumerate}
                \item[a)] Yesterday, Ronald \emph{(go)} \underline{\bf went} to school by car. (Ontem, Ronald foi para a escola de carro.)
                \item[b)] Sam \emph{(have)} \underline{\bf had} breakfast early today. (Sam tomou café da manhã cedo hoje.)
                \item[c)] Juliet \emph{(study)} \underline{\bf studied} in the University of California. (Juliet estudou na Universidade da Califórnia)
                \item[d)] I \emph{(eat)} \underline{\bf ate} lasagna last Sunday. (Eu comi lasanha no domingo passado.)
                \item[e)] My friend \emph{(arrive)} \underline{\bf arrived} a 10:00 p.m. last night. (Meu amigo chegou às 10h00 ontem à noite.) 
            \end{enumerate}

            \item[2.] Fill in the \textbf{Regular} verbs in parentheses into the sentences. Use the Simple Past Tense.
            \begin{enumerate}
                \item[a)] Lico \underline{\bf paid} soccer yesterday. (to play)
                \item[b)] Sarah \underline{\bf watched} a cartoon on TV. (to watch)
                \item[c)] They \underline{\bf carried} their books. (to carry)
                \item[d)] He \underline{\bf helped} his father last Saturday. (to help)
                \item[e)] John and Mary \underline{\bf cleaned} the house. (to clean)
            \end{enumerate}

            \item[3.] Fill in the \textbf{Irregular} verbs in parentheses into the sentences. Use the Simple Past Tense.
            \begin{enumerate}
                \item[a)] She \underline{\bf drove} a beautiful car last week. (to drive)
                \item[b)] Frank \underline{\bf ate} the whole pizza. (to eat)
                \item[c)] They \underline{\bf had} a very good teacher last year. (to have)
                \item[d)] The candidates \underline{\bf wrote} their opinion on the paper. (to write)
                \item[e)] He \underline{\bf sang} his favorite song. (to sing)
            \end{enumerate}

            \item[4.] Complete as sentenças com o Passado Simples dos verbos dados.
            \begin{enumerate}
                \item[a)] The cat \underline{\bf jumped} the fence. (to jump)
                \item[b)] The children \underline{\bf missed} the school bus. (to miss)
                \item[c)] You \underline{\bf wrote} a beautiful poem. (to write)
                \item[d)] My friends \underline{\bf tried} to understand Arabic language. (to try)
                \item[e)] They \underline{\bf studied} Math in their rooms. (to study)
            \end{enumerate}

            \pagebreak

            \item[5.] Traduza (para o português) as orações do exercício 4.
            \begin{enumerate}
                \item[a)] \_\_\_\_\_\_\_\_\_\_\_\_\_\_\_\_\_\_\_\_\_\_\_\_\_\_\_\_\_
                \item[b)] \_\_\_\_\_\_\_\_\_\_\_\_\_\_\_\_\_\_\_\_\_\_\_\_\_\_\_\_\_
                \item[c)] \_\_\_\_\_\_\_\_\_\_\_\_\_\_\_\_\_\_\_\_\_\_\_\_\_\_\_\_\_
                \item[d)] \_\_\_\_\_\_\_\_\_\_\_\_\_\_\_\_\_\_\_\_\_\_\_\_\_\_\_\_\_
                \item[e)] \_\_\_\_\_\_\_\_\_\_\_\_\_\_\_\_\_\_\_\_\_\_\_\_\_\_\_\_\_
            \end{enumerate}

            \item[6.] Agora, neque todas as orações do exercício 4.
            \begin{enumerate}
                \item[a)] \_\_\_\_\_\_\_\_\_\_\_\_\_\_\_\_\_\_\_\_\_\_\_\_\_\_\_\_\_
                \item[b)] \_\_\_\_\_\_\_\_\_\_\_\_\_\_\_\_\_\_\_\_\_\_\_\_\_\_\_\_\_
                \item[c)] \_\_\_\_\_\_\_\_\_\_\_\_\_\_\_\_\_\_\_\_\_\_\_\_\_\_\_\_\_
                \item[d)] \_\_\_\_\_\_\_\_\_\_\_\_\_\_\_\_\_\_\_\_\_\_\_\_\_\_\_\_\_
                \item[e)] \_\_\_\_\_\_\_\_\_\_\_\_\_\_\_\_\_\_\_\_\_\_\_\_\_\_\_\_\_
            \end{enumerate}
        \end{enumerate}
    \end{day}
    
    \begin{day}{28/04/2021}
        \title{3}{A estrutura do \emph{Present Perfect}}
        
        \vspace{3pt}
        
        \textbf{Para as afirmativas}: \colorx{green}{SUJEITO} + \colorx{red}{PRESENTE SIMPLES DO VERBO \emph{TO HAVE}} + PARTICÍPIO PASSADO DO VERBO PRINCIPAL. \\
        \colorx{green}{I} \colorx{red}{have} tried sushi before.
        
        \vspace{3pt}
        
        \textbf{Para as negativas}: \colorx{green}{SUJEITO} + \colorx{red}{PRESENTE SIMPLES DO VERBO \emph{TO HAVE}} + \colorx{red}{\emph{NOT}} + PARTICÍPIO PASSADO DO VERBO PRINCIPAL. \\
        \colorx{green}{I} \colorx{red}{have not} tried sushi before.
        
        \vspace{3pt}
        
        \textbf{Para as interrogativas}: \colorx{red}{PRESENTE SIMPLES DO VERBO \emph{TO HAVE}} + \colorx{green}{SUJEITO} + PARTICÍPIO PASSADO DO VERBO PRINCIPAL + ? \\
        \colorx{red}{Have} \colorx{green}{you} tried sushi before?
        
        \vspace{12pt}
        \noindent Utiliza-se o \emph{present perfect} para descreve ações que iniciam no passado e se estendem até o presente. \\
        Exemplos:
        
        \emph{I have \colorx{red}{studied} at school since \colorx{blue}{7 o'clock}}. (Eu tenho estudado na escola desde às sete horas)
        
        \emph{They have \colorx{red}{learned} about music since last year}. (Eles têm aprendido sobre música desde o ano passado)
        
        \vspace{12pt}
        \noindent Utiliza-se o \emph{present perfect} para descrever ações que vêm ocorrendo recentemente. \\
        Exemplos:
        
        \emph{They have \colorx{red}{been} sad recently}. (Eles têmestado tristes recentemente)
        
        \emph{We \colorx{red}{have} run every day in the park}. (Nós temos corrido todos os dias no parque)
        
        \vspace{12pt}
        \noindent Utiliza-se o \emph{present perfect} para descrever ações que acabaram de ocorrer. \\
        Exemplos:
        
        \emph{We \colorx{red}{have} just finished our work}. (Nós acabamos de terminar nosso trabalho)
        
        \emph{She \colorx{red}{has} just read the magazine}. (Ela acabou de ler a revista)
        
        \vspace{12pt}
        \noindent Utiliza-se o \emph{present perfect} para descrever ações que ocorreram em um momento indefinido do passado. \\
        Exemplos:
        
        \emph{You \colorx{red}{have} played video game for a long time}. (Você têm jogado vídeo game por um longo tempo)
        
        \emph{I \colorx{red}{have} cooked for you}. (Eu tenho cozinhado para vocês)
        
        \vspace{12pt}
        \noindent Formação das frases na negativa e na interrogativa:
        
        \noindent O negativo do \emph{present perfect} é formado pelo acréscimo do \textbf{"not"} na frase (forma contraída: \textbf{haven't/hasn't})
        
        \begin{enumerate}
            \item[1.] Supply the present perfect tense od the given: \\
            Example:
            
            I \underline{have} already \underline{written} the book. (write)
            
            \begin{enumerate}[nosep]
                \item[a)] He \underline{\bf has worked} here for three months. (work)
                \item[b)] They \underline{\bf have gone} to Australia many times. (go)
                \item[c)] \underline{\bf Have} you ever \underline{\bf went} to Paris? (be)
                \item[d)] My family \underline{\bf have visited} me recently. (visit)
                \item[e)] My parents \underline{\bf haven't arrived} yet. (not arrive)
            \end{enumerate}
        \end{enumerate}
    \end{day}
    
    \begin{day}{29/04/2021}
        \begin{enumerate}
            \item[3.] Fill in:
            \begin{enumerate}[nosep]
                \item[a)] What {\bf\underline{has it}} done? {\bf\underline{It drunk}} the milk. (drink)
                \item[b)] What's he done? {\bf\underline{He has cleaned}} the windows. (clean)
                \item[c)] What have they done? {\bf\underline{They have broken}} the window. (break)
            \end{enumerate}
        \end{enumerate}
        
        \begin{multicols}{2}
            \title{3}{A long way to travel}
        
            Alex and his mother were talking on the beach one day after a storm.
            
            "Look", Alex said. He pointed to a piece of round, vlue glass almost covered with sand. They dug around it and found a round, glass ball. It was bigger than a softball, but smaller thana soccer ball.
            
            "What is it?" Alex asked.
            
            \vfill\columnbreak
            
            \title{3}{Uma viagem distante}
            
            %Alex e sua mãe estavam conversando na praia após o temporal.
        
        \end{multicols}
    \end{day}
    
    \begin{day}{02/06/2021}
        \title{3}{Tag Questions}
        
        \begin{multicols}{2}
            \title{4}{English is the Official Language of 53 Countries?}
            
            400 million people around the world apeak English as their first language. Not only that, but English is liste as one of the official languages in more than a quarter of the countries in the world. That's a lot of new people you can communicate with just by improving one language, isn't it?
            \vfill\columnbreak
            \title{4}{Inglês é a língua oficial de 53 países?}
            
            400 milhões de pessoas em volta do mundo falam Inglês como língua primária. Não apenas isso, Inglês é listado como uma das línguas oficiais em mais de um quarto dos países no mundo. Isso é muita gente que você pode se comunicar aprendendo uma língua, não é?
        \end{multicols}
        
        \begin{multicols}{2}
            There are many, many reasons why learning a new language is a good idea, aren't there? It allows you to communicate with new people. It helps you to see things from a different perspective or get a deeper understanding of another culture. It helps you to become a better listener. It even has health benefits, as studies have shown that people who speak two or more languages have more active minds later in life.
            \vfill\columnbreak
            Há várias razões do porquê aprender uma nova língua é uma boa ideia, não é? Isso te possibilita a se comunicar com novas pessoas. Te ajuda a ver as coisas de perspectivas diferentes ou ter um entendimento mais aprofundado de uma outra cultura. Isso ajuda a se tornar um melhor ouvinte. E tem até benefícios de saúde, como estudantes têm mostrado que pessoas que falam duas ou mais línguas tem mentes mais ativas mais tarde na vida.
        \end{multicols}
        
        \begin{enumerate}
            \itemc[1.] What is the text about? \normalfont
            \begin{enumerate}[nosep]
                \item[a)] The reasons to avoid studying a second language.
                \itemc[\color{green}b)] The reasons someone should study a new language.
                \item[c)] The consequences of studying a third language.
            \end{enumerate}
            
            \itemc[2.] Mark the information that is not mentioned in the text. \normalfont
            \begin{enumerate}[nosep]
                \item[a)] Learning a new language can make you get any new culture better.
                \item[b)] You can talk with new people when learning a new language.
                \itemc[\color{green}c)] You become a better writer.
            \end{enumerate}
        \end{enumerate}
    \end{day}
    
    \begin{day}{10/06/2021}
        \title{3}{See the following examples}
        
        \colorx{blue}{English} \colorx{red}{is} easy, \colorx{red}{isn't} \colorx{blue}{it}?
        
        \colorx{blue}{She} \colorx{red}{can't} speak English, \colorx{red}{can} \colorx{blue}{she}?
        
        \colorx{red}{There are} new books in English, \colorx{red}{aren't there}?
        
        \colorx{blue}{Josh} \colorx{red}{speaks} English, \colorx{red}{doesn't} \colorx{blue}{he}?
        
        \title{3}{Fill with the tag questions}
        
        \begin{enumerate}
            \item[1.] I can't learn a new language, \textbf{\it can I}?
            \item[2.] They are studying, \textbf{\it aren't they}?
            \item[3.] She loves English, \textbf{\it doesn't she}?
            \item[4.] Donald needed help with the homework, \textbf{\it didn't he}?
        \end{enumerate}
        
        \begin{enumerate}
            \item[1.] Jess doesn't like to practice English, \uline{\hspace{2cm}}?
            \begin{enumerate}
                \item[a)] doesn't she
                \itemc[b)] does she
            \end{enumerate}
            
            \item[2.] It it dificult to read books in English, \uline{\hspace{2cm}}?
            \begin{enumerate}
                \item[a)] is it
                \itemc[b)] isn't it
            \end{enumerate}
        \end{enumerate}
        
        \begin{enumerate}
            \item[1.] Are you an adventurous person?
            \response Você é uma pessoa aventureira?
            
            \item[2.] Do you like extreme sports?
            \response Você gosta de esportes extremos/radicais?
            
            \item[3.] Look at he pictures. What sports are there?
            \response Olhe para as imagens. Quais esportes são esses? (Surf, Escalada)
        \end{enumerate}
        
        \begin{multicols}{2}
            \title{4}{Matt's Adventures}
            
            That's my cousin Matt. He's travelled all over the world. He loves extreme sports. Last July, he started taking surf lessons. The sport Matt really enjoys is rock climbing. He says he has met many new people with the same insane interests. He and his friends have had many adventures. They have broken many world records. But they have faced hundreds of problems together.
            \vfill\columnbreak
            \title{4}{Aventuras de Matt}
            
            Este é meu primo Matt. Ele viajou o mundo todo. Ele ama esportes radicais. No Julho passado ele começou a fazer aulas de surf. O esporte com que Matt mais se diverte é escalada. Ele disse que encontrou várias novas pessoas com o mesmo interesse. Ele e seus amigos tiveram muitas aventuras. Eles quebraram vários recordes mundiais, porém encararam centenas de problemas juntos.
        \end{multicols}
    \end{day}
    
    \begin{day}{2º Trimestre -- 17/06/2021}
        \begin{multicols}{2}
            \title{4}{Edinburgh}

            A visit to Scotland often begins in Edinburgh, the capital city of Scotland. Edinburgh is an old city with many important and interesting buildings. After London, Edinburgh is the second city for visitors in Britain.
            \vfill\columnbreak
            \title{4}{Edimburgo}
            
            Uma visita à Escócia geralmente começa em Edimburgo, sua capital. Edimbirgo é uma cidade antiga com muitas construções importantes e atraentes. Depois de London, Edimburgo é a segunda cidade para visitantes da Grã-Bretanha.
        \end{multicols}
        
        \begin{enumerate}
            \item[\bf 1.]\textbf{Qual é a capital da Escócia?}
            \response Edimburgo.
            
            \item[\bf 2.]\textbf{Edimburgo é muito visitada. Está em qual posição em relação a qual país?}
            \response Na escolha para visiantes, Edimburgo está uma posição abaixo de Londres.
        \end{enumerate}
        
        \begin{multicols}{2}
            If you come to Edinburg by train from the south, the first thing you see when you leave Waverley Station is Edinburgh Castle. It stands high over the city. Soldiers in kilts take visitors around and tell then the castle's story.
            \vfill\columnbreak
            Se você viaja para Edimburgo por trem vindo do sul, a primeira coisa que você vê ao deixar a Estação Waverley é o Castelo Edimburgo. Ele fica localizado no ponto mais alto da cidade. Soldados em kilts acompanham visitantes e lhes contam histórias sobre o castelo.
        \end{multicols}
        
        \begin{enumerate}
            \item[\bf 3.]\textbf{Essa roupa do soldado tem um nome. Qual é?}
            \response Kilt.
            
            \item[\bf 4.]\textbf{Qual é a primeira coisa que você vai ver ao sair da estação Waverlay?}
            \response O Castelo Edimburgo.
        \end{enumerate}
        
        \begin{multicols}{2}
            Edinburgh is a hilly city, but it is a good city to visit on foot. After the castle, you can visit more of the Old Town. Go down the Royal Mile to Holyroodhouse -- the Queen's home when she comes to Edinburgh. It is three hundreds years old.
            \vfill\columnbreak
            Edimburgo é uma cidade montanhosa, porém é uma boa cidade para visitar a pé. Depois do castelo você pode visitar mais da Old Town. Desça até a Royal Mile para Holyroodhouse -- a casa da Rainha quando ela vola para Edimburgo. Sua casa tem trezentos anos.
        \end{multicols}
        
        \begin{enumerate}
            \item[\bf 5.]\textbf{O que esse parágrafo está dizendo?}
            \response O parágrafo está dizendo: que Edimburgo é uma cidade montanhosa e a localização e idade da casa da rainha.
        \end{enumerate}
        
        \begin{multicols}{2}
            You can look at the shops on the Royal Mile or on Princes Street in the New Town. Some shops sell the famous Scottish tartans, and you can see the name of the family which goes with each tartan.
            \vfill\columnbreak
            Você pode ver as lojas na Royal Mile ou na Princess Street em New Town. Algumas lojas vendem o famoso tartã Escocês, e você pode ver o nome da família que acompanha cada tartã.
        \end{multicols}
        
        \begin{enumerate}
            \item[\bf 6.]\textbf{O parágrafo faz referênia a algumas lojas onde você pode encontrar o \emph{tartan}. Mas o que é isso? Veja o desenho, reflita, se possível converse com seus colegas e professor(a) e escreva sua opinião a respeito.}
            \response Tartã é a estampa quadriculada presente nos \emph{kilts}.
        \end{enumerate}
    \end{day}
    
    \begin{day}{24/06/2021}
        \begin{multicols} 2
            Near Princes Street is Charlotte Square, which is very beautiful. There is also the National Gallery of Scotland, with pictures from Scotland and from many other countries too.
            \vfill\columnbreak
            Próximo da Princes Street tem a Charlotte Square, que é muito bonito. Tem também a Galeria Nacional da Escócia, com imagens da cidade e de vários outros países também.
        \end{multicols}
        
        \begin{enumerate}
            \item[\bf 7.] \textbf{O que é a Charlotte Square mencionada no parágrafo?}
            \response Charlotte Square é uma praça.
        \end{enumerate}
        
        \begin{multicols} 2
            In August, you can visit the Edinburgh Festival, the biggest arts festival in the world with hundreds if different things to do and see.
            \vfill\columnbreak
            Em agosto você pode visitar o Festival de Edimburgo, o maior festival de artes do mundo, com centenas de coisas para ver.
        \end{multicols}
        
        \begin{enumerate}
            \item[\bf 7.] \textbf{O que acontece em agosto?}
            \response Em agosto acontece o Festival de Edimburgo, o maior festival de artes do mundo.
            
            \item[\bf 8.] \textbf{Você gosta desse tipo de festival?}
            \response Não.
        \end{enumerate}
        
        \title 3 {Preposição}
        
        \begin{enumerate}
            \item[1.] The following words will appear in the text you're about to read. Match them.
            
            \begin{tabular}{l l}
                a) priest   & (c) The legally recognized union of two people.  \\
                b) emperor  & (f) A written message.                           \\
                c) marriage & (a) An ordained minister of the Catholic church. \\
                d) death    & (d) The action of dying.                         \\
                e) blind    & (e) Unable to see.                               \\
                f) note     & (b) Someone who rules an empire.
            \end{tabular}
            
            \item[1.] As seguntes palavras estarão no texto que você está prestes a ler. Relacione-as.
            
            \begin{tabular}{l l}
                a) padre      & (c) A união legalmente reconhecida entre duas pessoas. \\
                b) emperador  & (f) Uma mensagem escrita.                              \\
                c) casasmento & (a) An ordained minister of the Catholic church.       \\
                d) morte      & (d) A ação de morrer.                                  \\
                e) cego       & (e) Impossibilidade de ver.                            \\
                f) nota       & (b) Alguém que rege um império.
            \end{tabular}
        \end{enumerate}
    \end{day}
    
    \begin{day}{30/06/2021}
        \title 3 {https://drive.google.com/file/d/1TfXBIylG3kLS-QQcSiZRun-CrKuUDIX4/view}
        
        \begin{enumerate}
            \item[\bf 1.] \textbf{De acordo com o text, responda:}
            \begin{enumerate}
                \item[a)] Onde e com quem Peter foi andar de bicicleta?
                \response No parque com is amigos.
                
                \item[b)] A que horas ele se levanta?
                \response 11 AM
                
                \item[c)] O que leu nas férias?
                \response Livros e quadrinhos.
                
                \item[d)] Onde ele almoçava?
                \response Nos restaurantes.
                
                \item[e)] O que ele fazia com sua irmã?
                \response Bebiam refigerante e comiam pipoca com chocolate.
            \end{enumerate}
            
            \item[\bf 2.] \textbf{Passe as frases abaixo para o passado.}
            \begin{enumerate}[nosep]
                \item[a)] She \underline{worked} last week. (work)
                \item[b)] Peter \underline{tried} acarajé for the first time. (try)
                \item[c)] Mark \underline{traveled} with your family to Salvador. (travel)
                \item[d)] Mary \underline{liked} the food in Bahia. (like)
            \end{enumerate}
            
            \item[\bf 3.] \textbf{Passe as frases abaixo para a forma negativa do passado. (didn't)}
            \begin{enumerate}
                \item[a)] John tried two shoes yesterday.
                \response John didn't tried two shoes yesterday.
                
                \item[b)] Paul studied German last week.
                \response Paul didn't studied German last week.
                
                \item[c)] Anne traveled to Italy last month.
                \response Anne didn't traveled to Italy last month.
                
                \item[d)] Joanne liked food in Minas.
                \response Joanne didn't liked food in Minas.
            \end{enumerate}
            
            \item[\bf 4.] \textbf{Passe as frases abaixo para a forma interrogativa do passado. (did)}
            \begin{enumerate}
                \item[a)] John tried two shoes yesterday.
                \response Did John tried two shoes yesterday?
                
                \item[b)] Paul studied German last week.
                \response Did Paul studied German last week?
                
                \item[c)] Anne traveled to Italy last month.
                \response Did Anne traveled to Italy last month?
                
                \item[d)] Joanne liked food in Minas.
                \response Did Joanne liked food in Minas?
            \end{enumerate}
            
            \item[\bf 5.] \textbf{Enumere as frases abaixo de 1 a 6 de acordo com a sequência dos acontecimento do texto.}
            
            Alberto Santos-Dumont (1873-1932) was born in Brazil and educated in Paris. He made his first balloon ascent in 1898 and soon after began constructing dirigible airships. In 1901 he won a Paris air race and international fame.
            
            Turning to hehave-than-air machines, Santos Dumont built his 14-Bis in 1906 and he returned to Brazil in 1916.
            
            \begin{enumerate}[nosep]
                \item[a)] \textbf{(4)} won a Paris air race and international fame.
                \item[b)] \textbf{(6)} returned to Brazil.
                \item[c)] \textbf{(5)} built his 14-Bis.
                \item[d)] \textbf{(3)} began constructing dirigible airships.
                \item[e)] \textbf{(2)} made his first balloon in 1898.
                \item[f)] \textbf{(1)} was born in Brazil.
            \end{enumerate}
        \end{enumerate}
    \end{day}
\end{document}






