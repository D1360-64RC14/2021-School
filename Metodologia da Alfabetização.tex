\documentclass{SchoolBook}

\begin{document}
    \begin{day}{28/04/2021}
        \title{3}{Ampliando Conhecimentos}
        
        \begin{enumerate}
            \item[1.] Observando esse vídeo o que foi exposto sobre a programação das atividades na Educação Infantil? Comente.
            %\response{Preocupação com o som das letras}
            \response{******************}
            
            \item[2.] Na Educação Infantil como deve ser o trabalho pedagógico? Comente.
            %\response{Desenvolver a consciência fonológica}
            \response{******************}
            
            \item[3.] O que é alfabetizar e letrar de acordo com o vídeo?
            %\response{Alfabetizar é a criança aprender a parte tecnica do conteúdo, e letrar é ela ter um maior conhecimento geral do significado}
            \response{******************}
        \end{enumerate}
    \end{day}
    
    \begin{day}{05/05/2021}
        \title{3}{Atividades para Sistematização de Conteúdo}
        
        \begin{enumerate}
            \item[1.] Letramento é uma palavra e um conceito recente, introduzido na linguagem da educação e das ciências linguíticas há pouco mais de duas décadas. Seu surgimento pode ser interpretado como decorrência da necessidade de configurar e nomear comportamentos e práticas sociais na área da:
            \begin{enumerate}[nosep]
                \itemc[a)] Leitura e Escrita;
                \item[b)] Leitura e Tradução;
                \item[c)] Matemática e Tectologia;
                \item[d)] Educação e Política;
                \item[e)] Ciência e Didática.
            \end{enumerate}
            
            \item[2.] Com base em seus conhecimentos podemos considerar que a leitura é:
            \begin{enumerate}[nosep]
                \item[a)] Um ato passivo e mecânico;
                \item[b)] Apenas uma decodificação de letras e números;
                \itemc[c)] Um ato global que promove o desenvolvimento intelectual.
            \end{enumerate}
            
            \item[3.] Quais dos métodos estudados baseiam-se nos conceitos tradicionais?
            \begin{enumerate}[nosep]
                \item[a)] Analíticos;
                \itemc[b)] Sintéticos;
                \item[c)] Global.
            \end{enumerate}
        \end{enumerate}
    \end{day}
    
    \begin{day}{12/05/2021}
        \begin{enumerate}
            \item[1.] Alfabetização é um processo complexo que envolve não apenas habilidades de codificar, mas se caracteriza como um processo ativo por meio do qual a criança constrói e reconstrói hipóteses. Dessa forma, assinale a opção CORRETA que expressa o conceito de alfabetização numa perspectiva mais ampla.
            \begin{enumerate}[nosep]
                \item[a)] Domínio da relação grafemas/fonemas, ou seja, decodificação e codificação.
                \item[b)] Decodificação dos sinais gráficos, transformando as letras em sons.
                \item[c)] Desenvolvimento da capacidade de codificar os sons da fala em sinais gráficos.
                \itemc[d)] Aprendizado da leitura e escrita, natueza e funcionamento do sistema de escrita.
                \item[e)] Apropriação de hipóteses de codificar e decodificar os sons em sinais gráficos.
            \end{enumerate}
            
            \item[2.] Assinale a alternativa correta. Com base em seus conhecimentos podemos considerar quea leitura é:
            \begin{enumerate}[nosep]
                \item[a)] Um ato passivo e mecânico.
                \item[b)] Apenas uma decodificação de letras e números.
                \itemc[c)] Um ato global que promove o desenvolvimento intelectual.
            \end{enumerate}
            
            \item[3.] Quais dos métodos estudados baseiam-se nos conceitos tradicionais? Assinale a alternativa correta.
            \begin{enumerate}[nosep]
                \item[a)] Analíticos.
                \itemc[b)] Sintéticos.
                \item[c)] Global.
            \end{enumerate}
        \end{enumerate}
    \end{day}
    
    \begin{day}{19/05/2021}
        \title{3}{Retomada de Conteúdos -- A Aquisição da Escrita}
        \title{4}{Assinale a alternativa correta.}
        
        \begin{enumerate}
            \item[1.] A comunicação verbal bem como os aspectos socioculturais do ser humano são externados psicofisicamente através da linguagem. Este é o meio de expressar e comunicar ideias, interpretar e usufruir as produções culturais nos mais diferentes contextos, atendendo a diferentes intensões e situações de comunicação. A linguagem é apresentada em diversos aspectos e situações. Sobre a linguagem não verbal podemos afirmar:
            \begin{enumerate}[nosep]
                \itemc[a)] Promove a comunicação de ideias, utilizando símbolos gráficos, imagens e, não necessariamente letras.
                \item[b)] Facilita aprendizagem da criança em diferentes contextos sociais.
                \item[c)] Promove a alfabetização através dos métodos de Ensino.
            \end{enumerate}
            
            \item[2.] As opções a seguir apresentam eixos necessários para a aquisição da língua escrita, à exceção de uma. Assinale-a.
            \begin{enumerate}[nosep]
                \item[a)] Compreensão e valorização da cultura escrita.
                \itemc[b)] Acesso somente a textos curtos e simples.
                \item[c)] Produção de textos escritos.
                \item[d)] Desenvolvimento da oralidade.
                \item[e)] Leitura.
            \end{enumerate}
            
            \item[3.] Defina Leitura.
            \response Leitura é o processo de decodificação de símbolos em informação.
        \end{enumerate}
        
        Elabore três objetivos específicos para aquisição da leitura.
        \response Decodificação dos símbolos gráficos, assim como a compreensão deles em conjunto entendendo a ideia passada.
    \end{day}
\end{document}
