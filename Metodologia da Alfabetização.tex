\documentclass{SchoolBook}

\begin{document}
    \begin{day}{28/04/2021}
        \title{3}{Ampliando Conhecimentos}
        
        \begin{enumerate}
            \item[1.] Observando esse vídeo o que foi exposto sobre a programação das atividades na Educação Infantil? Comente.
            %\response{Preocupação com o som das letras}
            \response{******************}
            
            \item[2.] Na Educação Infantil como deve ser o trabalho pedagógico? Comente.
            %\response{Desenvolver a consciência fonológica}
            \response{******************}
            
            \item[3.] O que é alfabetizar e letrar de acordo com o vídeo?
            %\response{Alfabetizar é a criança aprender a parte tecnica do conteúdo, e letrar é ela ter um maior conhecimento geral do significado}
            \response{******************}
        \end{enumerate}
    \end{day}
    
    \begin{day}{05/05/2021}
        \title{3}{Atividades para Sistematização de Conteúdo}
        
        \begin{enumerate}
            \item[1.] Letramento é uma palavra e um conceito recente, introduzido na linguagem da educação e das ciências linguíticas há pouco mais de duas décadas. Seu surgimento pode ser interpretado como decorrência da necessidade de configurar e nomear comportamentos e práticas sociais na área da:
            \begin{enumerate}[nosep]
                \itemc[a)] Leitura e Escrita;
                \item[b)] Leitura e Tradução;
                \item[c)] Matemática e Tectologia;
                \item[d)] Educação e Política;
                \item[e)] Ciência e Didática.
            \end{enumerate}
            
            \item[2.] Com base em seus conhecimentos podemos considerar que a leitura é:
            \begin{enumerate}[nosep]
                \item[a)] Um ato passivo e mecânico;
                \item[b)] Apenas uma decodificação de letras e números;
                \itemc[c)] Um ato global que promove o desenvolvimento intelectual.
            \end{enumerate}
            
            \item[3.] Quais dos métodos estudados baseiam-se nos conceitos tradicionais?
            \begin{enumerate}[nosep]
                \item[a)] Analíticos;
                \itemc[b)] Sintéticos;
                \item[c)] Global.
            \end{enumerate}
        \end{enumerate}
    \end{day}
    
    \begin{day}{12/05/2021}
        \begin{enumerate}
            \item[1.] Alfabetização é um processo complexo que envolve não apenas habilidades de codificar, mas se caracteriza como um processo ativo por meio do qual a criança constrói e reconstrói hipóteses. Dessa forma, assinale a opção CORRETA que expressa o conceito de alfabetização numa perspectiva mais ampla.
            \begin{enumerate}[nosep]
                \item[a)] Domínio da relação grafemas/fonemas, ou seja, decodificação e codificação.
                \item[b)] Decodificação dos sinais gráficos, transformando as letras em sons.
                \item[c)] Desenvolvimento da capacidade de codificar os sons da fala em sinais gráficos.
                \itemc[d)] Aprendizado da leitura e escrita, natueza e funcionamento do sistema de escrita.
                \item[e)] Apropriação de hipóteses de codificar e decodificar os sons em sinais gráficos.
            \end{enumerate}
            
            \item[2.] Assinale a alternativa correta. Com base em seus conhecimentos podemos considerar quea leitura é:
            \begin{enumerate}[nosep]
                \item[a)] Um ato passivo e mecânico.
                \item[b)] Apenas uma decodificação de letras e números.
                \itemc[c)] Um ato global que promove o desenvolvimento intelectual.
            \end{enumerate}
            
            \item[3.] Quais dos métodos estudados baseiam-se nos conceitos tradicionais? Assinale a alternativa correta.
            \begin{enumerate}[nosep]
                \item[a)] Analíticos.
                \itemc[b)] Sintéticos.
                \item[c)] Global.
            \end{enumerate}
        \end{enumerate}
    \end{day}
    
    \begin{day}{19/05/2021}
        \title{3}{Retomada de Conteúdos -- A Aquisição da Escrita}
        \title{4}{Assinale a alternativa correta.}
        
        \begin{enumerate}
            \item[1.] A comunicação verbal bem como os aspectos socioculturais do ser humano são externados psicofisicamente através da linguagem. Este é o meio de expressar e comunicar ideias, interpretar e usufruir as produções culturais nos mais diferentes contextos, atendendo a diferentes intensões e situações de comunicação. A linguagem é apresentada em diversos aspectos e situações. Sobre a linguagem não verbal podemos afirmar:
            \begin{enumerate}[nosep]
                \itemc[a)] Promove a comunicação de ideias, utilizando símbolos gráficos, imagens e, não necessariamente letras.
                \item[b)] Facilita aprendizagem da criança em diferentes contextos sociais.
                \item[c)] Promove a alfabetização através dos métodos de Ensino.
            \end{enumerate}
            
            \item[2.] As opções a seguir apresentam eixos necessários para a aquisição da língua escrita, à exceção de uma. Assinale-a.
            \begin{enumerate}[nosep]
                \item[a)] Compreensão e valorização da cultura escrita.
                \itemc[b)] Acesso somente a textos curtos e simples.
                \item[c)] Produção de textos escritos.
                \item[d)] Desenvolvimento da oralidade.
                \item[e)] Leitura.
            \end{enumerate}
            
            \item[3.] Defina Leitura.
            \response Leitura é o processo de decodificação de símbolos em informação.
        \end{enumerate}
        
        Elabore três objetivos específicos para aquisição da leitura.
        \response Decodificação dos símbolos gráficos, assim como a compreensão deles em conjunto entendendo a ideia passada.
    \end{day}
    
    \begin{day}{02/06/2021}
        Pesquise sobre as principais áreas de interesse da fonética:
        
        \begin{itemize}[nosep]
            \item Fonética Instrumental;
            \item Fonética Auditiva;
            \item Fonética Acústica.
        \end{itemize}
        
        \noindent\textbf{Fonética Instrumental}: é o estudo das propriedades físicas da fala. Compreende o estudo no apoio de instrumentos laboratoriais.
        
        \noindent\textbf{Fonética Auditiva}: é o estudo da percepção da fala.
        
        \noindent\textbf{Fonética Acústica}: tem a base de estudo nas propriedades físicas dos sons da fala, visando a transmissão de quem fala ao ouvinte.
    \end{day}
    
    \begin{day}{16/06/2021}
        \title{3}{Vídeo: Desafios para o Combate ao Analfabetismo no Brasil}
        
        \begin{enumerate}
            \item[1.] De acordo com o vídeo, o que é "analfabetismo Funcional"? Comente.
            \response Analfabetismo funcional é a situação onde o indivíduo não tem a capacidade de compreender textos simples ou realizar operações matemáticas básicas.
            
            \item[2.] Quais as causas do grande índice do analfabetismo no Brasil? Relate.
            \response Falta de infraestrutura, pobreza e evasão escolar.
            
            Muitos locais não tem uma infraestrutura adequada para receber uma escola, assim é necessário viajar todos os dias para ter acesso a educação, e muitas pessoas não tem condições disso.
            
            Muitas crianças tem que abandonar a escola por ter que se dedicar a outras tarefas, como trabalho, cuidados domésticos, ou falta de motivação e dificuldades de acesso.
            
            \item[3.] Qual o impacto do analfabetismo na sociedade? O que se pode apontar nesse sentido?
            \response A pessoa analfabeta, por não ter conhecimentos básicos ou não ter o ensino fundamental completo, tem muita dificuldade para encontrar trabalho, muito menos trabalhos que necessitem de de ensino superior.
            
            \item[4.] De acordo com o vídeo, qual a medida de intervenção para radicalizar esse problema no Brasil?
            \response É necessário um grande investimento em educação e projetos para aqueles que dizem ter "passado da idade de estudar". Estudos para aprimoramento do ensino também é muito bem-vindo.
            
            \item[5.] Quais os fatores que levam a evasão escolar? Comente.
            \response Trabalho infantil, cuidados domésticos, falta de motivação ou dificuldades de acesso.
            
            Por necessidades domésticas ou motivos familiares, crianças tem ou são obrigadas a parar os estudos para se dedicar a tarefas de casa ou trabalho para sustento próprio.
            
            Dificuldade de acesso ao ambiente escolar também é um fator para a criança deixar a escola, ela não vê futuro ou propósito no que está aprendendo. Isso combinado com dificuldade de acesso contribui muito para o abandono ao estudo, onde a criança tem que se deslocar um trajeto muito longo para chegar a escola ou não tem capacidade financeira para pagar meios de transporte.
        \end{enumerate} 
    \end{day}
    
    \begin{day}{29/07/2021}
        O objetivo de Paulo Freire era preparar cientificamente a população para o mercado de trabalho para atender às necessidades da época. Para isso, elaborou uma proposta de educação que se baseava nas necessidades da época e nas experiências da pessoa (do aluno).
    
        Preparar a população para o mercado de trabalho. Aprimorando o conhecimento. Para atender às necessidades da época. Para isso, elaborou proposta para educação de aultos: observando as necessidades da época e a experiencia do aluno. O conhecimento partia da realidade do aluno.
    \end{day}
    
    \begin{day}{04/08/2021}
        \begin{enumerate}
            \item[1.] Fale sobre os desafios da alfabetização e da educação de jovens e adultos no Brasil
            \response{}
            
            \begin{itemize}
                \item Instituições de ensino não querendo ofertar EJA ou cancelando programas do tipo;
                \item Despreparo dos professores para essa categoria de ensino;
                \item Sistema de educação não se adaptando às necessidades;
                \item Conflito entre obrigações do aluno e escola, levando ele a abandonar o ensino por falta de tempo;
                \item Pré-conceito da pessoa achando que "já passou da idade de estudar".
            \end{itemize}

            \item[2.] De que forma ocorre a educação de jovens e adultos?
                \response{}
                
            \item[3.] Fale sobre a história da educação Brasileira.
                \response{}
        \end{enumerate}
    \end{day}
\end{document}
