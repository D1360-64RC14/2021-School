\documentclass{SchoolBook}

\begin{document}
    \begin{day}{28/04/2021}
        \begin{enumerate}
            \item Explique o que seentende po "cise da ciência" no século XIX.
            %\response{Seria a descoberta de uma nova "verdade" sobre o assunto, contrariando o atual.}
            
            \response{No século XIX, a ciência enfrentou situações em que o paradigma da ciência moderna entrou em crise, isso significa que vrdades aceitas como definitivas foram colocadas em questão, isso caracterizou o que compreendemos como a crise da ciência moderna.}
            
            \item Analise as frases abaixo. De acordo com o critério de verificabilidade, qual das duas pode ser considerada uma proposição científica? Por quê?
            
            \begin{center}
                \emph{
                    "Existe petróleo em Marte?" \\
                    "Depois da morte, as almas evoluídas habitam em Marte"
                }
            \end{center}
            
            %\response{A primeira, pois é algo concreto, possível de reproduzir. Já a segunda é algo beirando o sobrenatural, assim sem conceitos sólidos.}
            \response{Atende ao critério de verificabilidade aquilo que se refere à experiência e pode ser verificado. Analisando as duas proposições, neste sentido, apenas a proposição "Existe petróleo em Marte" atende a este critério. Posso realizar uma expedição à marte a fim de constatar se é verdade. No entando, mesmo realizando uma expedição, não posso verificar se "Depois da morte, as almas evoluídas habitam em marte".}
        \end{enumerate}
    \end{day}
    
    \begin{day}{19/05/2021}
        
    \end{day}
\end{document}
